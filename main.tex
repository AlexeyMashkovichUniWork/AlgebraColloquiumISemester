\documentclass{article}

% Language setting
\usepackage[russian]{babel}

% Set page size and margins
% Replace `letterpaper' with `a4paper' for UK/EU standard size
\usepackage[a4paper,top=2cm,bottom=2cm,left=3cm,right=3cm,marginparwidth=1.75cm]{geometry}

% Useful packages
\usepackage{amsmath}
%\usepackage{unicode-math}
\usepackage{txfonts}
\usepackage{graphicx}
\usepackage{titlesec}
\usepackage[colorlinks=true, allcolors=blue]{hyperref}
\usepackage{lmodern}
\usepackage{tikz-cd}
\usepackage{textualicomma}
\usepackage{indentfirst}
\usepackage{chngcntr}
\usepackage{fdsymbol}

\usepackage{mathspec}
\setmathrm{XITS Math}

\usepackage[no-math]{fontspec}
\setmainfont{Roboto}

\counterwithin*{equation}{section}

\newcommand{\cgcd}{\mathrm{НОД}}
\newcommand{\clcm}{\mathrm{НОК}}
\newcommand{\divs}{~\vdots~}

\title{Билеты к коллоквиуму по алгебре I семестра, поток 25.Б7х-8х, матмех}
\author{Шмидт Р. А., Семёнов А. А., Кострикин А. И., Королёв А. В.}
\date{Версия 1.0, 25.10.2025}

\begin{document}
\maketitle

\tableofcontents

\sloppy

\newpage
\section{Понятие множества. Действия над множествами. Законы де Моргана}
\subsection{"Определение": Множество / элемент}
Первое из принципиально неопределяемых нами понятий - это понятие \textbf{множества}, или, говоря точнее, понятие \textbf{множества и его элемента}. По сути, множество и его элемент - это две стороны одного и того же понятия: множество полностью определяется своими элементами (состоит из своих элементов), т.е. для каждого объекта мы имеем хотя бы принципиальную возможность однозначно судить о том, является ли он элементом этого множества или нет.

\subsection{Определение: Принадлежность, подмножество / надмножество}

Если $x$ есть элемент множества $A$, то говорят также, что $x$ \textbf{принадлежит} $A$, или что $A$ \textbf{содержит} $x$, и пишут $x \in A$, в противном случае пишут $x \notin A$.

Пусть $(A, B)$ - пара множеств. Говорят, что $A$ есть \textbf{подмножество} множества $B$ ($A \subset B$), или что множество $B$ есть \textbf{надмножество} множества А ($B \supset A$), если каждый элемент множества $A$ является элементом множества $B$, т.е. $A \subset B \Leftrightarrow \forall x \in A : x \in B$. Говорят также, что $A$ \textbf{содержится} в $B$ или что $B$ \textbf{содержит} $A$; в последнем случае лишь контекст определяет, является $A$ подмножеством или элементом множества $B$.

\subsection{Определение: Объединение, пересечение, разность}

Пусть $(A, B)$ - пара множеств.

\textbf{Объединением} пары $(A, B)$ называется множество $A \cup B = \{x : x \in A \wedge x \in B \}$.

\textbf{Пересечением} пары $(A, B)$ называется множество $A \cap B = \{x : x \in A, x \in B \}$.

\textbf{Разностью} пары $(A, B)$ называется множество $A \setminus B = \{x : x \in A, x \notin B \}$.

\subsection{Законы де Моргана}

$\overline A = U \setminus A$, где $U$ - некое универсальное множество, содержащее $A$ и $B$.

\[\overline{A \cup B} = \overline A \cap \overline B \]
\[\overline{A \cap B} = \overline A \cup \overline B \]

\[\overline{\bigcup_{i\in I} A_i} = \bigcap_{i\in I}\overline{A_i}\]
\[\overline{\bigcap_{i\in I} A_i} = \bigcup_{i\in I}\overline{A_i}\]

\subsection{Определение: Пустое множество}
\textbf{Пустое множество} - это множество, не содержащее ни одного элемента. $\varnothing$ - обозначение пустого множества.

\subsection*{Список литературы}
Семёнов, Шмидт: стр. 10-11, 14

\newpage
\section{Мощность множества. Декартово произведение множеств. Счётные множества. Несчётные множества. Счётность множества рациональных чисел}

\subsection{Определение: Равномощность}

Пусть $A, B$ - пара множеств. Говорят, что $A$ \textbf{равномощно} $B$ (или $A$ и $B$ имеют одинаковую мощность, или мощности $A$ и $B$ равны и т.п.), если существует биекция $A \rightarrow B$ или $A = B = \varnothing$. При этом мы будем писать $|A| = |B|$, в противном случае $|A|\neq|B|$.

Только что введённое понятие задаёт в произвольном множестве множеств отношение - \textbf{отношение равномощности}. Ясно, что это отношение рефлексивно, симметрично и транзитивно.

\subsection{Сравнения мощностей}

Если существует инъекция $A$ в $B$ или $A = \varnothing$, то говорят, что \textbf{мощность} $A$ \textbf{не больше мощности} $B$, и пишут $|A| \leq |B|$. Если при этом $|A| \neq |B|$, то говорят, что мощность $A$ \textbf{меньше} мощности $B$, и пишут $|A| < |B|$. В частности, если $B \neq \varnothing$, то $|\varnothing| < |B|$.

\subsection{Определение: Конечность, бесконечность, счётность}

Мы будем называть множество $A$ \textbf{конечным} и писать $|A| < \infty$, если это множество равномощно начальному отрезку натурального ряда, отличному от всего этого ряда (в частности, пустое множество конечно). Если этот отрезок есть $\{1, 2, \dots, n\}$, то говорят, что $n$ есть число (или количество) элементов множества $A$, и пишут $|A| = n$ (для пустого множества $|\varnothing| = 0$). Бесконечным называют множество, не являющееся конечным.

Множество, равномощное множеству натуральных чисел, называется \textbf{счётным}.

\subsection{Декартово произведение множеств}

Пусть $(A, B)$ - пара множеств. \textbf{Декартовым произведением} этой пары называется множество $A \times B$ всех пар $(a, b)$, где $a \in A, b \in B$: \[
	A \times B = \{(a, b):a \in A, b \in B\}
\]

\subsection{Предложение 1: Объединение конечного и счётного множества счётно}
\subsubsection*{Доказательство}
Пусть $A = \{a_1, a_2, \dots, a_m\}, B = \{b_1, b_2, \dots, b_n, \dots\}$. Предположим, что $A \cap B = \varnothing$. В противном случае в объединении $A \cup B$ вместо $A$ нужно будет рассмотреть множество $A \setminus B$, которое будет конечным. Итак, $A \cup B = \{a_1, a_2, \dots, a_m, b_1, b_2, \dots, b_n, \dots\}$.

Чтобы установить счётность, $A \cup B$ достаточно представить в виде его последовательности $c_1, c_2, \dots, c_n, \dots$. Действительно, в этом случае каждому элементу множества $A \cup B$ будет сопоставлено одно и только одно натуральное число - его номер в этой последовательности. Тем самым будет установлено, что $A \cup B$ равномощно $\mathbb{N}$. То есть, \[c_1 = a_1, c_2 = a_2, \dots, c_m = a_m, c_{m+1} = b_1, c_{m+2} = b_2, \dots\].


\subsection{Предложение 2: Объединениие конечного числа счётных множеств счётно}
\subsubsection*{Доказательство}

Рассмотрим счётные множества $A$ и $B$. Предположим, что $A \cap B = \varnothing$.

Если это не так, то воспользуемся тем, что $A \cup B = (A \setminus B) \cup B$. Множество $A \setminus B$ будет либо конечным, либо счётным. В первом случае $A \cup B$ будет счётным по предыдущему предложению, во втором - докажем ниже.

Пусть ${A = \{a_1, a_2, \dots, a_n, \dots \}}, {B = \{b_1, b_2, \dots, b_n, \dots\}}$. Тогда ${A \cup B = \{a_1, b_1, a_2, b_2, \dots, a_n, b_n, \dots\}}$. Для доказательства счётности множества $A \cup B$ достаточно представить его в виде последовательности $c_1, c_2, \dots, c_n, \dots$. Сделаем это так: $c_1 = a_1, c_2 = b_1, c_3 = a_2, c_4 = b_2, c_5 = a_3, c_6 = b_3, \dots, c_{2n-1} = a_n, c_{2n} = b_n, \dots$.

Взаимно однозначное соответствие между элементами объединения и множеством натуральных чисел установлено. Доказательство счётности $A_1 \cup A_2 \cup A_3 \cup \dots \cup A_{n-1} \cup A_n$ проводится индукцией по числу множеств.

\subsection{Теорема 1: Объединение счётного числа счётных множеств счётно}
\subsubsection*{Доказательство}
Рассмотрим счётное число счётных множеств $A_1, A_2, A_3, \dots, A_{n-1}, A_n, \dots$. Будем предполагать, что они попарно не пересекаются; в противном случае, вместо них можон рассмотреть не более чем счётные (т.е. конечное или счётные) множества ${A_1, A_2 \setminus A_1, A_3 \setminus (A_1 \cup A_2), \dots}$, объединение которых совпадает с объединением исходных множеств ${A_1 \cup A_2 \cup ... \cup A_{n-1} \cup A_n \cup \dots}$. Итак, пусть \[
	A_1 = \{a_{11}; a_{12}; a_{13}; a_{14}; \dots; a_{1n}; \dots \}, \]\[
	A_2 = \{a_{21}; a_{22}; a_{23}; a_{24}; \dots; a_{2n}; \dots \}, \]\[
	A_3 = \{a_{31}; a_{32}; a_{33}; a_{34}; \dots; a_{3n}; \dots \}, \]\[
	A_4 = \{a_{41}; a_{42}; a_{43}; a_{44}; \dots; a_{4n}; \dots \}.
\]	

Установим порядок пересчёта элементов таблицы. Первый номер дадим такому элементу, у которого сумма индексов равна $2$ (это $a_{11}$). 

Следующие два номера сопоставим тем элементам, у которых сумма индексов равна $3$, причём договоримся, что номера таким элементам будем присваивать в порядке убывания первых индексов; таким образом, $2 \leftrightarrow a_{21}, 3 \leftrightarrow a_{12}$. 

Номера 4-6 получат элементы с суммой индексов равной $4$, которые пересчитываются далее в порядке убывания первых индексов, и так далее.

Иначе говоря, мы пересчитываем элементы таблицы, двигаясь последовательно по её диагоналям, начиная с левого верхнего угла. Очевидно, что при этом каждому элементу таблицы сопоставляется единственное натуральное число, а каждому натуральному числу - единственный элемент таблицы. Правило пересчёта установлено. Это и доказывает счётность объединения $A_1, A_2, A_3, \dots, A_{n-1}, A_n, \dots$.

\subsection{Теорема 2: Множество рациональных чисел счётно}
\subsubsection*{Доказательство}

Докажем сначала счётность $\mathbb{Q}_+$ (множество \textit{положительных} рациональных чисел). Представим это множество как объединение счётного числа счётных множеств. Пусть \[
	A_1 = \{\frac{1}{1}; \frac{1}{2}; \frac{1}{3}; \frac{1}{4}; \cdots; \frac{1}{n}; \cdots\}, \]\[
	A_2 = \{\frac{2}{1}; \frac{2}{2}; \frac{2}{3}; \frac{2}{4}; \cdots; \frac{2}{n}; \cdots\}, \]\[
	A_3 = \{\frac{3}{1}; \frac{3}{2}; \frac{3}{3}; \frac{3}{4}; \cdots; \frac{3}{n}; \cdots\}, \]\[
	\cdots, \]\[
	A_m = \{\frac{m}{1}; \frac{m}{2}; \frac{m}{3}; \frac{m}{4}; \cdots; \frac{m}{n}; \cdots\}, \]\[
	\cdots
\]

Очевидно, что $\mathbb{Q}_+ = A_1 \cup A_2 \cup A_3 \cup \dots \cup A_m \cup \dots$. Тогда $\mathbb{Q}_+$ счётно как объединение счётного числа счётных множеств (теорема 1).

Ясно, что $\mathbb{Q}$ равномощно $\mathbb{Q}_+$, что значит, что $\mathbb{Q}$ также счётно. Тогда $\mathbb{Q}$ счётно как объединение двух счётных и одного конечного множества $\{0\}$ (предложения 1 и 2).

\subsection{Теорема 3: Множество точек промежутка $(0; 1)$ не является счётным}

\subsubsection*{Доказательство}

Пусть $A = \{x \in \mathbb{R} : 0 < x < 1\}$. Предположим, что $A$ счётно, а значит, все его элементы можно пронумеровать, т.е. представить в виде последовательности: $\{a_1, a_2, a_3, \dots, a_n, \dots\}$.

Каждый член последовательности - это число из промежутка $(0; 1)$ и поэтому его можно представить в виде десятичной дроби $0, \alpha_1, \alpha_2, \alpha_3, \dots, \alpha_n, \dots$, где $\alpha_1, \alpha_2, \alpha_3, \dots, \alpha_n, \dots$ - цифры от $0$ до $9$.

Условимся конечные десятичные дроби записывать как периодические с периодом $0$. Например, число $0,27$ запишем так: $0,27000\dots$. Кроме того, договоримся не использовать запись чисел в виде бесконечной дорби с периодом $9$, чтобы избежать неоднозначной записи. Например, будем писать $0,23000\dots$ вместо $0,22999\dots = 0,22(9)$. Тогда для любого числа из промежутка его запись в виде $0, \alpha_1, \alpha_2, \alpha_3, \dots, \alpha_n, \dots$ будет единственной.

Итак, пусть \begin{gather*}
a_1 = 0, \alpha_{1, 1}, \alpha_{1, 2}, \alpha_{1, 3}, \alpha_{1, 4}, \dots, \alpha_{1, n}, \dots \\
a_2 = 0, \alpha_{2, 1}, \alpha_{2, 2}, \alpha_{2, 3}, \alpha_{2, 4}, \dots, \alpha_{2, n}, \dots \\
\dots \\
a_{n-1} = 0, \alpha_{n-1, 1}, \alpha_{n-1, 2}, \alpha_{n-1, 3}, \alpha_{n-1, 4}, \dots, \alpha_{n-1, n}, \dots \\
a_n = 0, \alpha_{n, 1}, \alpha_{n, 2}, \alpha_{n, 3}, \alpha_{n, 4}, \dots, \alpha_{n, n}, \dots \\
\dots \\
\end{gather*}

У цифры $\alpha_{i, j}$ первый индекс $i$ - это номер элемента $a_i$ в последовательности, а второй индекс $j$ - номер знака после запятой.

Укажем такое число $b$, которое с одной стороны принадлежит промежутку $(0; 1)$, а с другой - не совпадает ни с одним членом последовательности $\{a_1, a_2, a_3, \dots, a_n, \dots \}$.

Пусть $b = 0, \beta_1, \beta_2, \beta_3, \dots, \beta_n, \dots$, где \[
	\beta_1 \neq \alpha_{1, 1} \Rightarrow b \neq \alpha_1 \]\[
	\beta_2 \neq \alpha_{2, 2} \Rightarrow b \neq \alpha_2 \]\[
	\beta_3 \neq \alpha_{3, 3} \Rightarrow b \neq \alpha_3 \]\[
	\cdots \]\[
	\beta_n \neq \alpha_{n, n} \Rightarrow b \neq \alpha_n \]\[
	\cdots \]

При выборе $i$-го знака после запятой в числе $b$ будем избегать цифр $0$ и $9$, чтобы исключить возможность получения бесконечной дроби с периодом $9$, а также совпадения числа $b$ с $0 = 0,(0)$ или $1 = 0,(9)$.

Таким образом, число $b$ будет принадлежать промежутку $(0; 1)$, но не будет совпадать ни с одним из членов последовательности $\{a_1, a_2, a_3, \dots, a_n, \dots \}$.

Следовательно, множество точек промежутка $(0; 1)$ не является счётным.

\subsection{Определение: Мощность континуума}
Будем говорить, что множества, равномощные множеству точек промежутка $(0; 1)$, имеют мощность континуума.

Чтобы проверить, что множество всех вещественных чисел имеет мощность континуума, достаточно установить взаимно однозначное соответствие между точками всей числовой оси и точками $(0; 1)$. Это можно сделать многими способами, например \[ (0; 1) \ni x \mapsto tg[(x - 0,5)\frac{\pi}{2}] \]	

\subsection*{Список литературы}
Семёнов, Шмидт: стр. 16, 25-26

\newpage
\section{Отображения. Инъективность. Сюръективность. Биективность. Примеры. Композиция отображений}
\subsection{Определение: Отображение, множество всех отображений, начало, конец}
Бинарное отношение $f$ из $A$ в $B$ называется \textbf{отображением} $A$ в $B$, если ${\forall x \in A \enspace \exists!y \in B : xfy}$, то есть, если каждому элементу множества $A$ соответствует единственный элемент множества $B$.

\textbf{Множество всех отображений} $A$ в $B$ обозначается $B^A$.

Тот факт, что $f$ есть отображение $A$ в $B$ записывается так: $f: A \rightarrow B$; мы будем называть $A$ \textbf{началом} и $B$ - \textbf{концом} отображения $f$.

\subsection{Определение: Образ, прообраз, полный образ, полный прообраз}
Элемент множества $B$, соответствующий элементу $x \in A$ при отображении $f: A \rightarrow B$, называют \textbf{образом элемента} $x$ (при отображении $f$) и обозначают $f(x)$.

Если $y$ - образ элемента $x$, то каждый такой $x$ называют \textbf{прообразом элемента} $x$.

Пусть $f: A \rightarrow B, A' \subset A, B' \subset B$. Множество $f(A') = \{f(x) : x \in A'\}$, состоящее из образов всех элементов множества $A'$, называется \textbf{образом множества} $A'$.

Образ $f(A)$ всего начала отображения $f$ называется \textbf{образом отображения} $f$ и обозначается $Im~f$.

Аналогично, множество $f^{-1}(B') = \{x \in A : f(x) \in B'\}$ всех прообразов элементов множества $B'$ называется \textbf{полным прообразом} множества $B'$. Важно не путать прообраз элемента $y$ с его полным прообразом: прообраз элемента $y$ есть \textit{элемент} множества $A'$, и прообразов одного элемента может быть сколько угодно, в том числе ни одного, а \textit{полный} прообраз элемента $y$ есть \textit{подмножество} множества $A$, состоящее из всех прообразов элемента $y$.

\subsection{Инъективность, сюръективность, биективность}
\subsubsection*{Определение}
Отображение $f: A \rightarrow B$ называется: \begin{itemize}
	\item \textbf{инъективным} (или \textbf{инъекций}), если $\forall a, b \in A \quad a \neq b \Rightarrow f(a) \neq f(b)$;
	\item \textbf{сюръективным} (или \textbf{сюръекцией}), если $Im~f = B$;
	\item \textbf{биективным} (или \textbf{биекцией}), если оно инъективно и сюръективно.
\end{itemize}

\subsubsection*{Примеры}
\begin{itemize}
	\item $f: \mathbb{R}_+ \rightarrow \mathbb{R} \quad f(x) = x^2$ - инъекция
	\item $f: \mathbb{R} \rightarrow \mathbb{R} \quad f(x) = x^2$ не является инъекцией ($f(-2) = f(2) = 4$)
	\item $f(x) = \sin x \quad f: \mathbb{R} \rightarrow [-1; 1]$ - сюръекция
	\item $f(x) = \sin x \quad f: \mathbb{R} \rightarrow \mathbb{R}$ не является сюръекцией
	\item $f: \mathbb{R} \rightarrow \mathbb{R} \quad f(x) = 2x+1$ - биекция
	\item $f: \mathbb{R} \rightarrow \mathbb{R} \quad f(x) = x^3$ - биекция
\end{itemize}

\subsection{Определение: композиция}
Пусть $f: B \rightarrow C$ и $g: A \rightarrow B$ - пара отображений. \textbf{Произведением} этой пары называется отображение $fg: A \rightarrow C$ такое, что $\forall x \in A \quad fg(x) = f(g(x))$. Произведение отображений называют также \textbf{композицией} или \textbf{суперпозицией}, обозначают также $f \circ g$.

\subsection{Предложение: композиция ассоциативна}
То есть, если есть три отображения - $
	h: U \rightarrow V; \quad
	g: V \rightarrow W; \quad
	f: W \rightarrow T
$, то ${f(gh) = (fg)h}$

\subsubsection*{Доказательство}
\[
	\forall u \in U \quad (f(gh))(u) = f(gh(u)) = f(g(h(u))) = (fg)(h(u)) = ((fg)h)(u)
\]

\subsection{Определение: Ограничение, срезка, сужение}
Пусть $f: A \rightarrow B, A' \subset A, B' \subset B, Im~f \subset B'$.

\textbf{Ограничением} отображения $f$ на $A'$ называется отображение ${f|}_{A'}: A' \rightarrow B$ такое, что $\forall X \in A' {f|}_{A'}(x) = f(x)$.

\textbf{Срезкой} отображения $f$ на $B'$ будем называть отображение $A$ в $B'$, действующее на элементы из $A$ так же, как $f$. Стандартного обозначения срезка не имеет.

\textbf{Сужением} отображения $f$ будем называть срезку ограничения $f$, т.е. отображение $A'$ в $B'$, где $A' \subset A, f(A') \subset B' \subset B$.

\subsection{Список литературы}

Семёнов, Шмидт: стр. 18-19, 21-23, 26-27

Кострикин I: стр. 37-38


\newpage
\section{Обратное отображение. Теорема об обратном отображении}
\subsection{Определение: Тождественное отображение}
Пусть $A$ - произвольное множество. \textbf{Тождественным отображением} в $A$ называется отображение $id_A: A \rightarrow A$ такое, что $\forall x \in A \quad id_A(x) = x$.

\subsection{Свойство тождественного отображения}
Пусть $f: A \rightarrow B$. Тогда $f \circ id_A = id_B \circ f = f$:
\shorthandoff{"}%
\[
\begin{tikzcd}
B &B \lar["id_B"'] &A \ar[ll, bend left, "id_B \circ f"] \lar["f"'] &A \lar["id_A"] \ar[ll, bend right, "f \circ id_A"'] 
\end{tikzcd}
\]
\shorthandon{"}%

Из диаграммы видно, что $id_B \circ f$ и $f \circ id_A$ определены, их начала и концы совпадают с началом и концом $f$. \[
	\forall x \in A \quad f \circ id_A(x) = f(id_A(x)) = f(x); id_B \circ f(x) = id_B(f(x)) = f(x)
\] т.е. графики отображений $f$, $f \circ id_A$, $id_B \circ f$ также совпадают. Совпадение начал, концов и графиков и означает совпадение этих отображений.

\subsection{Определение: Обратные отображения}
Пусть $f: A \rightarrow B$. Отображение $g$ называется:
\begin{itemize}
\item \textbf{Левым обратным} отображению $f$, если $gf = id_A$;
\item \textbf{Правым обратным} отображению $f$, если $fg = id_B$;
\item \textbf{Обратным} отображению $f$, если $g$ есть левое \textit{и} правое обратное.
\end{itemize}

\subsubsection*{Замечание}
Ясно, что $g$ - левое обратное $f$ тогда и только тогда, когда $f$ - правое обратное $g$.

\subsection{Предложении о единственности обратного отображения}

Если у отображения $f$ существует левое и правое обратные отображения, то они совпадают, и в этом случае у $f$ имеется единственное левое обратное, оно же единственное правое обратное, оно же единственное обратное отображение. В частности, если у отображения существует обратное, то оно единственно.

\subsubsection*{Доказательство}
Пусть $g$ - левое, $h$ - правое обратные отображению $f : A \rightarrow B$. Тогда \[
gfh = \begin{cases}
(gf)h = id_A \circ h = h \\
g(fh) = g \circ id_B = g
\end{cases}
\] откуда $g = h$. Если $g'$ - также левое обратное $f$, то, по доказанному, $g' = h$, т.е. $g' = g$. Аналогично показывается единственность правого обратного. Последнее утверждение предложения очевидно.

Обратное $f$ отображение обычно обозначается $f^{-1}$.

\subsection{Определение: Обратимость}

Отображение называется:
\begin{itemize}
\item \textbf{Обратимым слева}, если существует левое обратное ему отображение;
\item \textbf{Обратимым справа}, если существует правое обратное ему отображение;
\item \textbf{Обратимым}, если существует обратное ему отображение.
\end{itemize}

\subsection{Теорема: Критерии обратимости}
Отображение
\begin{enumerate}
\item \textbf{обратимо слева} тогда и только тогда, когда оно \textbf{инъективно},
\item \textbf{обратимо справа} тогда и только тогда, когда оно \textbf{сюръективно},
\item \textbf{обратимо} тогда и только тогда, когда оно \textbf{инъективно}.
\end{enumerate}

\subsubsection*{Доказательство}
\begin{enumerate}
\item Пусть $f: A \rightarrow B$ обратимо слева, т.е. $\exists g : B \rightarrow A$ такое, что $gf = id_A$. Пусть $x, x' \in A$ и $f(x) = f(x')$. Тогда $x = id_A(x) = (gf)(x) = g(f(x)) = g(f(x')) = (gf)(x') = id_A(x') = x'$, то есть $f$ - инъекция.

Обратно, пусть $f$ инъективно. Тогда $\forall y \in Im~f \subset B \quad \exists ! x(y) \in A : f(x) = y$. Положим $g(y) = x(y)$ для $y \in Im~f$, а для $y \in B \setminus Im~f$ значение $g(y)$ выберем в $A$ совершенно произвольно. Тогда, очевидно, получим $gf = id_A$.

\item Пусть $f : A \rightarrow B$ обратимо справа, т.е. $\exists g : B \rightarrow A$ такое, что $fg = id_B$. Тогда $\forall y \in B \quad y = id_B(y) = (fg)(y) = f(g(y))$, т. е. $f$ - сюръекция.

Обратно, пусть $f$ сюръективно. У каждого элемента $y \in B$ есть его полный прообраз при отображении $f$. Эти прообразы у разных элементов $y_1, y_2 \in B$ не пересекаются. В качестве $g(y)$ выберем любой $x$ из полного прообраза $y$, т.е. $x \in f^{-1}(y) \subset A$. Тогда, очевидно, $fg = id_B$.

\item Очевидным образом следует из пп. 1 и 2.
\end{enumerate}

\subsection{Следствие теоремы}
Очевидным образом вытекают из теоремы и замечания следующие утверждения:

\begin{itemize}
\item \textbf{Левое обратное} какому-либо отображение (необходимо инъективному) \textbf{сюръективно}.
\item \textbf{Правое обратное} какому-либо отображение (необходимо сюръективному) \textbf{инъективно}.
\item \textbf{Обратное} какому-либо отображение (необходимо биективному) \textbf{биективно}.
\end{itemize}


\subsection*{Список литературы}
Семёнов, Шмидт: стр. 23-25


\newpage
\section{Отношения. Эквивалентность. Разбиение множества на классы эквивалентности}

\subsection{Определение: Бинарное отношение, график, соответствие}
Пусть $(A, B)$ - пара непустых множеств. \textbf{Бинарным отношением} в этой паре (или из $A$ в $B$) называется тройка $(A, B, R)$, где $R \subset A \times B$.

Если $\rho = (A, B, R)$ - бинарное отношение, то $R$ называется его \textbf{графиком}. Вместо ${(a, b) \in R}$ пишут $a \rho b$ и говорят, что $a$ \textbf{соответствует} $b$ (по отношению $\rho$), или что $a$ и $b$ находятся в этом отношении (соответствии). Если $A = B$, то говорят, что $\rho$ - отношение в $A$.

\subsection{Определение: Рефлексивность, симметричность, транзитивность}
Отношение $\rho$ в множестве $A$ называется
\begin{itemize}
\item \textbf{Рефлексивным}, если $\forall a \in A \quad a \rho a$;
\item \textbf{Симметричным}, если $a \rho b \Rightarrow b \rho a$;
\item \textbf{Транзитивным}, если $a \rho b, b \rho c \Rightarrow a \rho c$.
\end{itemize}

\subsection{Определение: Эквивалентность, класс эквивалентности, представитель}
Отношение в некотором множестве называется \textbf{отношением эквивалентности}, или \textbf{эквивалентностью}, если оно рефлексивно, симметрично и транзитивно.

Пусть $\sim$ обозначает некоторую эквивалентность в $A$, и $a \in A$. Множество $\tilde{a} = \{x \in A : x \sim a\}$ называется \textbf{классом эквивалентности} $\sim$, \textbf{порождённым} $a$ (ясно, что $\tilde{a} \subset A$), его элементы называются \textbf{представителями} этого класса. В силу рефлексивности отношения $\forall a \in A \quad a \in \tilde{a}$.

\subsection{Немного свойств классов эквивалентности}

Пусть $a' \in \tilde{a}$, т.е. $a' \sim a$. Тогда $\forall x \in \tilde{a'} \quad x \sim a' \sim a$, и по транзитивности $x \sim a$. Значит, $\tilde{a'} \subset \tilde{a}$. В силу симметричности $a \sim a'$ и таким же рассуждением получим $\tilde{a} \subset \tilde{a'}$. Итак, если $a' \in \tilde{a}$, то $\tilde{a'} = \tilde{a}$, т.е. \textbf{каждый класс эквивалентности порождается любым своим элементом} (представителем).

Пусть $\tilde{a}, \tilde{b}$ - пара классов эквивалентности $\sim$. Если $\tilde{a} \cap \tilde{b} \neq \varnothing$, то $\exists x \in \tilde{a} \cap \tilde{b}$, т.е. $x \in \tilde{a}, x \in \tilde{b}$, а потому $\tilde{x} = \tilde{a}, \tilde{x} = \tilde{b}$, то есть $\tilde{a} = \tilde{b}$. Таким образом, \textbf{классы эквивалентности либо не пересекаются, либо совпадают}.

\subsection{Определение: Фактор-множество, разбиение / дизъюнктное объединение}

Множество всех классов эквивалентности $\sim$ в $A$ называется \textbf{фактор-множеством} множества $A$ по этой эквивалентности и обозначается $A/\sim$.

$A$ есть объединение множества всех классов данной эквивалентности: $A = \underset{\tilde{a} \in A/\sim}{\bigcup} \tilde{a}$. При этом если $\tilde{a} \neq \tilde{b}$, то $\tilde{a} \cap \tilde{b} = \varnothing$. Представление некоторого множества как объединение его попарно непересекающихся непустых подмножеств называется \textbf{разбиением} этого множества (говорят также, что что \textbf{дизъюнктное объединение}). Таким образом, всякая эквивалентности в некотором множестве определяет некоторое разбиение этого множества, а именно разбиение его на классы этой эквивалентности.

Обратно, пусть $A = \underset{i}{\bigcup}A_i$ - произвольное разбиение множества $A$. Зададим в $A$ отношение $\sim$ следующим образом: $a \sim a' \Leftrightarrow \exists i : a \in A_i, a' \in A_i$. Тогда $\sim$ - эквивалентность, а все $A_i$ - все её классы.

Таким образом, имеется взаимно однозначное соответствие между множеством всех эквивалентностей в $A$ и множеством всех разбиений этого множества. Другими словами, понятия эквивалентности и разбиения по существу равнозначны, и всё равно, говорить ли о некоторой эквивалентности или о некотором разбиении.

\subsection*{Список литературы}
Семёнов, Шмидт: стр. 17-18, 32-34


\newpage
\section{Перестановки. Разложение перестановки в произведение независимых циклов и транспозиций}
\subsection{Определение: Перестановка}
Пусть $\Omega$ - конечное множество из $n$ элементов. Удобно считать, что $\Omega = \{1, 2, \dots, n\}$, поскольку природа его элементов для нас несущественна. Элементы множества $S_n = S(\Omega)$ всех взаимно однозначных преобразований $\Omega \rightarrow \Omega$ называются \textbf{перестановками} и обычно обозначаются строчными буквами греческого алфавита. Лишь за единичным преобразованием $e = e_\Omega$ сохранилась буква латинского алфавита.

\subsection{Развёрнуная запись}
В развёрнутой и наглядной форме произвольную перестановку $\pi : i \rightarrow \pi(i), i = 1, 2, \dots, n$ изображают в виде \[
\pi = \left(\begin{matrix}1 & 2 & \cdots & n \\ i_1 & i_2 & \cdots & i_n \end{matrix} \right)
\] полностью указывая все образы: \[
\pi: 
\begin{tikzcd}[column sep=0.5, row sep=small]
1 \dar[mapsto]	& 2 \dar[mapsto]	& \cdots	& n \dar[mapsto] \\
i_1 			& i_2 				& \cdots 	& i_n
\end{tikzcd}
\] где $i_k = \pi(k) \quad (k = 1, 2, \dots, n)$ - переставленные символы $1, 2, \dots, n$.

\subsection{Композиция перестановок}
Перестановки перемножаются в соответствии с общим правилом композиции отображений: $(\sigma\tau)(i) = \sigma(\tau(i))$. Например, для перестановки $\sigma = \left(\begin{matrix}1 & 2 & 3 & 4 \\ 2 & 3 & 4 & 1\end{matrix}\right), \tau = \left(\begin{matrix}1 & 2 & 3 & 4 \\ 4 & 3 & 2 & 1\end{matrix}\right)$ имеем \[
\sigma\tau = \left(\begin{matrix}1 & 2 & 3 & 4 \\ 2 & 3 & 4 & 1\end{matrix}\right)\left(\begin{matrix}1 & 2 & 3 & 4 \\ 4 & 3 & 2 & 1\end{matrix}\right) = \begin{tikzcd}[column sep=0.5, row sep=small]
1 \dar[mapsto]	& 2 \dar[mapsto]	& 3 \dar[mapsto]	& 4 \dar[mapsto] \\
4 \dar[mapsto]	& 3 \dar[mapsto]	& 2 \dar[mapsto]	& 1 \dar[mapsto] \\
1				& 4 				& 3 				& 2
\end{tikzcd} = \left(\begin{matrix}1 & 2 & 3 & 4 \\ 1 & 4 & 3 & 2\end{matrix}\right)
\]

Ясно, что умножение перестановок подчиняется правилам композиции отображений:
\begin{enumerate}
\item Умножение ассоциативно.
\item $S_n$ обладает единичным элементом: $\pi e = e\pi = \pi \quad \forall \pi \in S_n$
\item Для каждой перестановки $\pi \in S_n$ существует обратная перестановка $\pi^{-1}: \pi \pi^{-1} = \pi^{-1}\pi = e$.
\end{enumerate}
Эти 3 свойства позволяют говорить о группе $S_n$. Множество $S_n$, рассматриваемое вместе с естественной операцией умножения его элементов (композицией перестановок) называется симметрической группой степени $n$.

Ясно, что порядок группы $\mathrm{Card}~S_n = |S_n| = n!$.

\subsection{Краткая запись, Опеределение: цикл, произведение циклов}

Перестановка $\sigma$, кратко записываемая в виде $\sigma = (\begin{matrix}1&2&3&4\end{matrix})$, или, что то же самое, в виде $\sigma = (\begin{matrix}2&3&4&1\end{matrix}) = (\begin{matrix}3&4&1&2\end{matrix}) = (\begin{matrix}4&1&2&3\end{matrix})$, называется \textbf{циклом} длины 4, а перестановка $\tau = (\begin{matrix}1&4\end{matrix})(\begin{matrix}2&3\end{matrix})$ - \textbf{произведением} двух независимых (непересекающихся) \textbf{циклов} $(\begin{matrix}1&4\end{matrix})$ и $(\begin{matrix}2&3\end{matrix})$ длины 2.
\[
\sigma = \begin{tikzcd}
					& 2 \dlar[bend right]	&						\\
3 \drar[bend right]	&						& 1 \ular[bend right]	\\
					& 4 \urar[bend right]	&
\end{tikzcd}, \quad \tau = \begin{tikzcd}
4 \rar[bend right=80] & 1 \lar[bend right=80] & 3 \rar[bend right=80] & 2 \lar [bend right=80]
\end{tikzcd}
\]

Заметим, что $\sigma^2 = (\begin{matrix}1&3\end{matrix})(\begin{matrix}2&4\end{matrix}), \sigma^4 = (\sigma^2)^2 = e, \tau^2 = e$.

Степень перестановки $\pi^s$ определяется по индукции \[
\pi^s = \begin{cases}\begin{array}{ll}
\pi(\pi^{s-1}) \quad				& s > 0 \\
e									& s = 0 \\
\pi^{-1}((\pi^{-1})^{(-s-1)}) \quad	& s < 0
\end{array}
\end{cases}
\]

Ясно, что $\pi^s \pi^t = \pi^{s+t} = \pi^t \pi^s, \quad s, t \in \mathbb{Z}$.

\subsection{Определение: порядок перестановки, эквивалентность, орбита}

Так как $|\Omega| < \infty$, то для каждой перестановки $\pi \in S_n$ найдётся однозначно определённое натуральное число $q = q(\pi)$ такое, что все различные степени содержатся во множестве $\left<\pi\right> = \{e, \pi, \dots, \pi^{q-1}\}$ и $\pi^q = e$. Это число $q$ называется \textbf{порядком перестановки} $\pi$.

Две точки (два элемента) $i, j \in \Omega$ назовём $\mathbf{\pi}$\textbf{-эквивалентными}, если $j = \pi^s(i)$ для некоторого $s \in \mathbb{Z}$.

Так как \[i = \pi^0(i)\]
\[j = \pi^s(i) \Rightarrow i = \pi^{-s}(j)\]
\[j = \pi^s(i), k = \pi^t(j) \Rightarrow k = \pi^{s+t}(i)\]
то мы имеем дело с рефлексивным, симметричным и транзитивным отношением на $\Omega$.

В сответствии с общим свойством эквивалентности получаем разбиение \begin{equation}
\label{eq:6_1}
\Omega = \Omega_1 \cup \dots \cup \Omega_p
\end{equation} множества $\Omega$ на попарно непересекающиеся классы $\Omega_1, \Omega_2, \dots, \Omega_p$, которые принято называть $\mathbf{\pi}$\textbf{-орбитами}. Каждая точка $i \in \Omega$ принадлежит в точности одной орбите, и если $i \in \Omega_k$, то $\Omega_k$ состоит из образов точки $i$ при действии степеней элемента $\pi$: 
\[i, \pi(i), \pi^2(i), \dots, \pi^{l_k-1}(i)\]
здесь $l_k = \Omega_k$ - длина $\pi$-орбиты $\Omega_k$.

Ясно, что $l_k \leq q = \mathrm{Card}~\pi, \enspace \pi^{l_k}(i) = i$, причём $l_k$ - наименьшее число, обладающее этим свойством.

Положив
\[\pi_k = (\begin{matrix}\pi(i) &\pi^2(i) &\dots &\pi^{l_k-1}(i)\end{matrix}) = \left(\begin{matrix}
i &\pi(i) &\dots &\pi^{l_k-2}(i) \\
\pi(i) &\pi^2(i) &\dots &\pi^{l_k-1}(i)
\end{matrix}\right)\]
мы придём как раз к перестановке, называемой циклом длины $l_k$.

Цикл $\pi_k$ оставляет на месте все точки из множества $\Omega \setminus \Omega_k$, а $\pi(j) = \pi_k(j)$ для любой точки $j \in \Omega_k$. Это даёт нам основание называть $\pi_s, \pi_t, s \neq t$ независимыми, или непересекающимися, циклами. Так как $\pi^{l_k}(i) = i$ для $i \in \Omega_k$, то $\pi_k^{l_k} = e$.

Итак, с разбиением \eqref{eq:6_1} ассоциируется разложение перестановки $\pi$ в произведение \begin{equation}
\label{eq:6_2}
\pi = \pi_1\pi_2\dots\pi_p
\end{equation} где все циклы перестановочны. Если цикл $\pi_k = (i)$ имеет длину 1, то он действует как единичная перестановка. Естественно такие циклы в произведении \eqref{eq:6_2} опускать.

Например, перестановку
\[\pi = \left(\begin{matrix}
1&2&3&4&5&6&7&8 \\
2&3&4&5&1&7&6&8
\end{matrix}\right) \in S_8\]
мы запишем в виде
\[\pi = (\begin{matrix} 1&2&3&4&5\end{matrix})(\begin{matrix}6&7\end{matrix})(8) =
(\begin{matrix} 1&2&3&4&5\end{matrix})(\begin{matrix}6&7\end{matrix})\]

\subsection{Теорема о единственности разложения перестановок}

(Примечание редактора: формулировка в конце)

Итак,
\begin{equation}
\label{eq:6_3}
\pi = \pi_1\pi_2\dots\pi_m, l_k > 1, 1 \leq k \leq m
\end{equation}

Пусть наряду с разложением \eqref{eq:6_3} мы имеем ещё одно разложение $\pi = \alpha_1\alpha_2\dots\alpha_r$ в произведение независимых циклов, и пусть $i$ - символ, не остающийся на месте при действии $\pi$.

Тогда $\pi_s(i) \neq i, \alpha_t(i) \neq i$ для одного (и только одного) из циклов $\pi_1, \pi_2, \dots, \pi_m$ и одного из $\alpha_1, \alpha_2, \dots, alpha_r$.

Имеем
\[ \pi_s(i) = \pi(i) = \alpha_t(i) \]

Если мы уже знаем, что \begin{equation}
\label{eq:6_4}
\pi_s^k(i) = pi^k(i) = \alpha_t^k(i)
\end{equation} то (индукционный переход), применяя к этим неравенствам перестановку $\pi$ и используя перестановочность $\pi$ с $\pi_s^k$ и с $\alpha_t^k(i)$, получаем
\[ \pi\pi_s^k = \pi^{k+1}(i) = \pi\alpha_t^k(i) \]
откуда $\pi_s^k\pi = \pi^{k+1}(i) = \alpha_t^k\pi(i)$ и, наконец,
\[ \pi_s^{k+1}(i) = \pi^{k+1}(i) = \alpha_t^{k+1}(i) \]

Значит, равенства \eqref{eq:6_4} справедливы при любом $k = 0, 1, 2, \dots$. Но цикл однозначно определяется действием его степеней на любой символ, который не остаётся на месте. Следовательно, $\pi_s = \alpha_t$. Дальше очевидная индукция по $m$ или $r$.

Итак, доказана теорема: \textit{Каждая перестановка $\pi \neq e$ в $S_n$ является произведением независимых циклов длины $\geq 2$. Это разложение в произведение определено однозначно с точностью до порядка следования циклов.}

\subsection{Определение: транспозиция}
\textbf{Цикл длины 2} называется \textbf{транспозицией}.

\subsection{Следствие теоремы}
\textit{Каждая перестановка является произведением транспозиций.}

В самом деле, в силу теоремы достаточно записать в виде произведения транспозиций каждый из циклов. Это можно сделать, например, так:
\[(\begin{matrix}1&2&\dots&l-1&l\end{matrix}) = (\begin{matrix}1 & l\end{matrix})(\begin{matrix}1 & l-1\end{matrix})\dots(\begin{matrix}1 & 3\end{matrix})(\begin{matrix}1 & 2\end{matrix})\]

Например,
\[(\begin{matrix}1&2&3&4&5\end{matrix}) = \left(\begin{matrix}
1&2&3&4&5 \\
2&3&4&5&1
\end{matrix}\right) =\]
\[
\begin{tikzcd}[column sep=0.5, row sep=small]
1 \dar[mapsto]	& 2 \dar[mapsto]	& 3 \dar[mapsto]	& 4 \dar[mapsto]	& 5 \dar[mapsto]	\\
2 \dar[mapsto]	& 1 \dar[mapsto]	& 3 \dar[mapsto]	& 4 \dar[mapsto]	& 5 \dar[mapsto]	\\
2 \dar[mapsto]	& 3 \dar[mapsto]	& 1 \dar[mapsto]	& 4 \dar[mapsto]	& 5 \dar[mapsto]	\\
2 \dar[mapsto]	& 3 \dar[mapsto]	& 4 \dar[mapsto]	& 1 \dar[mapsto]	& 5 \dar[mapsto]	\\
2				& 3					& 4					& 5					& 1
\end{tikzcd}
 = (\begin{matrix}1 & 5\end{matrix})(\begin{matrix}1 & 4\end{matrix})(\begin{matrix}1 & 3\end{matrix})(\begin{matrix}1 & 2\end{matrix})\]
 
\subsection{Список литературы}
Кострикин I: стр. 50-55


\newpage
\section{Понятие группы. Пример симметричной и знакопеременной группы. Их порядки}
\subsection{Определение: Группа, абелева группа}
\textbf{Группой} называется структура, заданная на некотором множестве $G$ одним всюду определённым внутренним действием $*$, удовлетворяющий следующим аксиомам:
\begin{enumerate}
    \item $*$ ассоциативно,
    \item В $G$ существует нейтральный относительно $*$ элемент,
    \item Каждый элемент из $G$ обратим относительно $*$.
\end{enumerate}

Если действие в группе коммутативно, то группа называется \textbf{коммутативной}, или \textbf{абелевой}.

\subsection{Определение: Порядок группы, единица / ноль}

Наличие в $G$ действия означает в частности означает, что $G \neq \varnothing$. Мощность $|G|$ множества $G$ называют \textbf{порядком группы} $G$. \textbf{Нейтральный элемент} называется, в зависимости от записи, \textbf{единицей} (при мультипликативной записи) или \textbf{нулём} (при аддитивной записи) группы $G$.

\subsection{Определение: Симметрическая группа степени $n$}
Пусть $\Omega = \{1, 2, \dots, n\}$ - конечное множество элементов. Множество всех взаимно однозначных преобразований $\Omega \rightarrow \Omega$ образуют группу перестановок $n$ элементов, которую называют \textbf{симметрической группой степени $n$} и обозначают $S_n$.

Ясно, что порядок этой группы $|S_n| = n!$.
\subsection{Теорема, Определение: знак}
Пусть $\pi$ - перестановка из $S_n$, \begin{equation}
\label{eq:7_1}
\pi = \tau_1\tau_2\dots\tau_k
\end{equation} - произвольное разложение $\pi$ в произведение транспозиций.

Тогда число $\varepsilon_\pi = (-1)^k$, называемое \textbf{знаком} $\pi$ (иначе: сигнатурой или чётностью), полностью определяется перестановкой $\pi$ и не зависит от способа разложения \eqref{eq:7_1}, т.е. чётность целого числа $k$ для данной перестановки $\pi$ всегда одна и та же. Кроме того, 
\[ \varepsilon_{\alpha\beta} = \varepsilon_\alpha\varepsilon_\beta \]
для всех $\alpha, \beta \in S_n$.
\subsubsection{Доказательство независимости знака от способа разложения}

Предположим, что наряду с \eqref{eq:7_1} мы имеем также разложение \[
    \pi = \tau'_1\tau'_2\dots\tau'_{k'},
\] причём чётности $k$ и $k'$ различны. Тогда целое число $k + k'$ нечётно. Так как $(\tau'_S)^2 = e$, то, следовательно, умножая справа обе части равенства \[
    \tau_1\tau_2\dots\tau_k = \tau'_1\tau'_2\dots\tau'_{k'}
\] на $\tau'_{k'},\dots,\tau'_2,\tau'_1$, получим $\tau_1\tau_2\dots\tau_k\tau'_{k'}\dots\tau'_2\tau'_1 = e$.

Мы свели нашу задачу к следующей. Пусть \begin{equation}
\label{eq:7_2}
e = \sigma_1\sigma_2\dots\sigma_{m-1}\sigma_m,\quad m>0
\end{equation} - запись единичной перестановки в виде произведения $m > 0$ транспозиций. Нужно показать, что $m$ - обязательно чётное число.

Нам нужно обосновать переход от записи \eqref{eq:7_2} к записи $e$ в виде $m-2$ транспозиций.

Пусть $s, 1\leq s\leq n$ - любое фиксированное натуральное число, входящее в одну из транспозиций. Для определённости считаем, что \[
e = \sigma_1\dots\sigma_{p-1}\sigma_p\sigma_{p+1}\dots\sigma_m
\] где $\sigma_p = (\begin{matrix}s&t\end{matrix})$, а $\sigma_{p+1},\dots,\sigma_m$ не содержат $s$.

Для $\sigma_{p-1}$ имеются 4 возможности:
\begin{enumerate}
    \item[а)] $\sigma_{p-1} = (\begin{matrix}s&t\end{matrix})$; тогда отрезок $(\begin{matrix}s&t\end{matrix})(\begin{matrix}s&t\end{matrix})$ из записи $e$ удаляется и мы приходим к $m - 2$ транспозициям;
    \item[б)] $\sigma_{p-1} = (\begin{matrix}s&r\end{matrix}), r \neq s, t$; здесь \[
    \sigma_{p-1}\sigma_p = (\begin{matrix}s&r\end{matrix})(\begin{matrix}s&t\end{matrix}) = (\begin{matrix}s&t\end{matrix})(\begin{matrix}r&t\end{matrix}) \]\[
    \begin{tikzcd}[column sep=0.5, row sep=small]
        r \dar[mapsto]	& s \dar[mapsto]	& t \dar[mapsto]	\\
        r \dar[mapsto]	& t \dar[mapsto]	& s \dar[mapsto]	\\
        s				& t					& r
    \end{tikzcd}
    \qquad 
    \begin{tikzcd}[column sep=0.5, row sep=small]
		r \dar[mapsto]	& s \dar[mapsto]	& t \dar[mapsto]	\\
		t \dar[mapsto]	& s \dar[mapsto]	& r \dar[mapsto]	\\
		s				& t					& r
    \end{tikzcd}\] и мы сдвинули вхождение $S$ на одну позицию влево, не изменив $m$;
    \item[в)] $\sigma_{p-1}$ = $(\begin{matrix}t&r\end{matrix}), r \neq s, t$; здесь \[
        \sigma_{p-1}\sigma_p = (\begin{matrix}t&r\end{matrix})(\begin{matrix}s&t\end{matrix}) = (\begin{matrix}s&r\end{matrix})(\begin{matrix}t&r\end{matrix})
    \]\[
    \begin{tikzcd}[column sep=0.5, row sep=small]
        r \dar[mapsto]	& s \dar[mapsto]	& t \dar[mapsto]	\\
        r \dar[mapsto]	& t \dar[mapsto]	& s \dar[mapsto]	\\
        t 				& r 				& s
    \end{tikzcd}
    \qquad
    \begin{tikzcd}[column sep=0.5, row sep=small]
    	r \dar[mapsto]	& s \dar[mapsto]	& t \dar[mapsto]	\\
    	t \dar[mapsto]	& s \dar[mapsto]	& r \dar[mapsto]	\\
    	t 				& r 				& s
    \end{tikzcd} \] и снова, как и в случае б), произошёл сдвиг $S$ влево без изменения $m$;
    \item[г)] $\sigma_{p-1}\sigma_p = (\begin{matrix}q&r\end{matrix})(\begin{matrix}s&t\end{matrix}) = (\begin{matrix}s&t\end{matrix})(\begin{matrix}q&r\end{matrix})$.
\end{enumerate}
В случае а) наша цель достигнута. В случаях б)-г) повторяем процесс, сдвигая $S$ на одну позицию влево. В конечном счёте мы придём либо к случаю а), либо к экстремальному случаю, когда $e = \sigma'_1\sigma'_2\dots\sigma'_m$, причём $\sigma'_1 = (\begin{matrix}s&t'\end{matrix})$ и $s$ не имеет вхождений в $\sigma'_2\dots\sigma'_m$. Значит, $\forall k>1 \quad \sigma'_k(s) = s$ и $s = e(s) = \sigma_1'(s) = t' \neq s$, что \textbf{является противоречием}. Полученное противоречне доказывает утвреждение об инвариантности $\varepsilon_\pi$, потому что, продолжая этот спуск от $m$ к $m-2$ множителей, мы пришли бы к нечётном $m$ к одной транспозиции $\tau$. Но, очевидно, $e \neq \tau$.

\subsubsection{Доказательство мультипликативности}
Если $\alpha = \tau_1\tau_2\dots\tau_k, \beta=\tau_{k+1}\dots\tau_{k+l}$, то $\alpha\beta = \tau_1\tau_2\dots\tau_k\tau_{k+1}\dots\tau_{k+l}$ и \[\varepsilon_\alpha = (-1)^k, \varepsilon_\beta = (-1)^l, \varepsilon_{\alpha\beta} = (-1)^k(-1)^l = (-1)^{k+l} = \varepsilon_\alpha\varepsilon_\beta\].

\subsection{Определение: Чётная, нечётная перестановка}
Перестановка $\pi \in S_n$ называется \textbf{чётной}, если $\varepsilon_\pi = 1$, и \textbf{нечётной}, если $\varepsilon_\pi = -1$.

\subsection{Следствия}
Пусть перестановка $\pi \in S_n$ разложена в произведение независимых циклов длин $l_1, l_2, \dots, l_m$. Тогда \[\varepsilon_\pi = (-1)^{\left(\sum_{k = 1}^{m}(l_k) - 1\right)}\]
Действительно, $\varepsilon_\pi = \varepsilon_{\pi_1\dots\pi_m} = \varepsilon_{\pi_1}\dots\varepsilon_{\pi_m}$.

Кроме того, \[\varepsilon_{\pi_k} = (-1)^{l_k-1}\] поскольку $\pi_k$ записывается в виде произведения $l_k-1$ транспозиций:
\[(\begin{matrix}1&2&\dots & {l_k-1} & {l_k}\end{matrix}) = (\begin{matrix}1 & {l_k}\end{matrix})(\begin{matrix}1 & {l_k-1}\end{matrix})\dots(\begin{matrix}1 & 3\end{matrix})(\begin{matrix}1 & 2\end{matrix})\]
Окончательно,
\[\varepsilon_\pi = (-1)^{l_1-1}\dots(-1)^(l_m-1) = (-1)^{\left(\sum_{k = 1}^{m}(l_k) - 1\right)}\]

\subsection{Количество чётных / нечётных перестановок}
Запишем $S_n$ в виде объединения $S_n = A_n \cup \overline{A_n}$, где $A_n = \left\{\pi \in S_n : \varepsilon_\pi = 1\right\}$ - множество всех чётных перестановок, $\overline{A_n} = S_n \setminus A_n$ - множество нечётных перестановок. Пусть $\tau = (\begin{matrix}i & j\end{matrix})$ - любая транспозиция. Отображение $S_n$ в себя, определённое правилом $L_\tau : \pi \rightarrow \tau_\pi$, очевидно, биективно. ($L_\tau^2$ - единичное отображение, $L_\tau^{-1} = L_\tau$).

Заметим, что 
\[L_\tau(A_n) = \overline{A_n}, \quad L_\tau(\overline{A_n}) = A_n\]

Значит, число чётных перестановок совпадает с числом нечётных перестановок, откуда
\[|A_n| = \frac{1}{2}|S_n| = \frac{n!}{2}\]

\subsection{Определение: знакопеременная группа}

Поскольку произведение двух чётных перестановок является чётной перестановкой, единичное преобразование является чётной перестановкой, и обратное преобразование для любой чётной перестановки также является чётной перестановкой (так как раскладывается в произведение транспозиций, или проще: $1 = \varepsilon(e) = \varepsilon(\varsigma\varsigma^{-1}) \Rightarrow \varepsilon(\varsigma) = \varepsilon(\varsigma^{-1})$, так как $\varepsilon = \pm1$), множество чётных перестановок тоже образует группу.

Она называется \textbf{знакопеременной группой} степени $n$ и обозначается обычно $A_n$. Порядок знакопеременной группы степени $n$ равен $\frac{1}{2}|S_n| = \frac{n!}{2}$.

\subsection*{Список литературы}
Семёнов, Шмидт: стр. 59

Кострикин I: стр. 56-58


\newpage
\section{Понятие кольца. Примеры колец. Область целостности. Кольцо классов вычетов и область целостности}
\subsection{Определение: Кольцо, коммутативное к., к. с единицей, тело, поле, к. с нулевым умножением}
\textbf{Кольцом} называется структура $(A, +, \cdot)$, заданная на некотором множестве $A$ парой всюду определённых внутренних действий - сложением и умножением, - удовлетворяющих следующим аксиомам:
\begin{enumerate}
\item Сложение ассоциативно;
\item Сложение коммутативно;
\item Существует $0$ - нейтральный элемент по сложению: $\forall a \in A \quad a+0 = 0+a$;
\item Каждый элемент из $A$ обратим по сложению: $\forall a \in A \quad \exists -a : a + (-a) = 0$;
\item Умножение ассоциативно;
\item Умножение дистрибутивно относительно сложения.
\end{enumerate}

Кроме аксиом 1-6 действия в кольце могут удовлетворять и другим требованиям:
\begin{enumerate}
\item[7.] Умножение коммутативно;
\item[8.] Существует 1 - нейтральный элемент по умножению: $\forall a \in A \quad a \neq 0 \Rightarrow a \cdot 1 = 1 \cdot a = a$, причём $1 \neq 0$;
\item[9.] Каждый отличный от нуля элемент кольца обратим.
\end{enumerate}

Кольцо, дополнительно удовлетворяющее:
\begin{itemize}
\item аксиоме \textbf{7} называется \textbf{коммутативным кольцом};
\item аксиоме \textbf{8} - \textbf{кольцом с единицей};
\item аксиомам \textbf{7 и 8} - \textbf{коммутативным кольцом с единицей};
\item аксиомам \textbf{8 и 9} - \textbf{телом};
\item аксиомам \textbf{7, 8 и 9} - \textbf{полем}.
\end{itemize}

\subsubsection{Примеры колец, Определение: Кольцо с нулевым умножением}
Базовые примеры: $\mathbb{Z}, \mathbb{Q}, \mathbb{R}$.

Множество $2\mathbb{Z}$ всех чётных чисел с обычными действиями есть кольцо коммутативное, но без единицы. $A$ - абелева группа, а умножение задано так: $\forall a, b \in A \quad ab = 0$. Коммутативное кольцо без единицы, так называемое \textbf{кольцо с нулевым умножением}.

\subsection{Определение: Делитель нуля}
Элемент $a \in R$ называется \textbf{делителем нуля}, если существует такое $x \in R, x \neq 0$, что $ax = 0$.

\subsection{Определение: Кольцо без делителей нуля}
Кольцо $R$ называется \textbf{кольцом без делителей нуля}, если 0 - единственный делитель нуля в $R$.

\subsection{Определение: Область целостности}
Коммутативное кольцо с единицей без делителей нуля называется \textbf{областью целостности}.

\subsection{Предложение о взаимной простоте, обратимости и делителе нуля}
Пусть $a, m \in \mathbb{Z}$.
\begin{enumerate}
\item Если $a, m$ взаимно просты, то $\overline a$ обратим в $\mathbb Z_m$.
\item Если $a, m$ не взаимно просты, то $\overline a$ - делитель нуля в $\mathbb Z_m$.
\end{enumerate}

\subsubsection*{Доказательство}
\begin{enumerate}
\item По теореме о линейном представлении НОД $1 = au + m\upsilon$ при некоторых $u, \upsilon \in \mathbb Z$. Отсюда $\overline 1 = \bar a \bar u + \bar m \bar \upsilon = \bar a \bar u$, т. е. $\overline u = \overline a^{-1}$, т. е. $\overline a$ обратим.
\item Пусть $d = \cgcd(a, m) \not \sim 1, a = a_1d, m = m_1d$. Тогда $m_1$ не делится на $m$, т. е. $\overline{m_1} \neq 0$, но $\bar a \overline{m_1} = \overline{am_1} = \overline{a_1dm_1} = \overline{a_1m} = \overline{0}$, т. е. $\overline a$ - делитель нуля.
\end{enumerate}

\subsection{Следствие}
\begin{enumerate}
\item Если $p \in \mathbb N$ просто, то $\mathbb Z_p$ - поле.
\item Если $m \in \mathbb N$ не просто и не равно 1, то $\mathbb Z_m$ содержит ненулевые делители нуля.
\end{enumerate}
\subsubsection*{Доказательство}
\begin{enumerate}
\item Пусть $a \in \mathbb Z$. Поскольку $p$ просто, $a \divs p$ или $a, p$ взаимно просты (свойства простых чисел), т. е. или $\overline a = 0$ или $\overline a$ обратим. Таким образом, каждый ненулевой класс из $\mathbb Z_p$ обратим, т. е. $\mathbb Z_p$ - поле.
\item Поскольку $m$ не просто и не равно один, $m$ имеет нетривиальный делитель $a$, $\textsf{НОД}(m, a) = a \not \sim 1$, т. е. $m, a$ не взаимно просты. $\overline a$ делит ноль в $\mathbb Z_m$. $a \not \sim m$, т. е. $a$ не делится на $m$, и потому $\overline a \neq 0$.
\end{enumerate}

Таким образом, при $m \geq 2$ кольцо $\mathbb Z_m$ либо суть поле, либо не есть даже область целостности.

При $m = 0$ кольцо $\mathbb Z_m = \mathbb Z_0$, которое можно отождествить с $\mathbb Z$, есть область целостности, но не поле.

При $m = 1$ кольцо $\mathbb Z_1 = \{0\}$ не имеет ненулевых делителей нуля, но не есть область целостности, ибо в нём $\overline 1 = \overline 0$.

Поля $\mathbb Z_p$ (при простых $p$) являют собой примеры конечных полей (их называют \textbf{полями Галуа} и обозначают $F_p$ или $GF(p)$).

\subsection*{Список литературы}

Семёнов, Шмидт: стр. 52-53, 55, 59, 79-80

\newpage
\section{Свойства делимости в кольце целых чисел. Теорема о делении с остатком}
\subsection{Определение: Делимость}
Пусть $a, b \in \mathbb Z$. Говорят, что $a$ делится на $b$ ($a \divs b$), или что $b$ делит $a$ ($b~\vert~a$), если существует такое $x \in \mathbb Z$, что $a = bx$.

\subsection{Свойства делимости}
\begin{enumerate}
\item $0$ делится на любое число: $\forall a \in \mathbb Z \quad 0 \divs a$.
\item $\forall a \in \mathbb Z \quad a \divs 1$ и $a \divs a$.
\item Делимость транзитивна: $a \divs b \divs c \Rightarrow a \divs c$.
\item Если все члены семейства $(a_i)$ делятся на $b$, то и любая линейная комбинация этого семейства делится на $b$.
\end{enumerate}

\subsection{Теорема о делениии целых чисел с остатком}
\[\forall a, b \in \mathbb Z, b \neq 0 \quad \exists! q, r \in \mathbb Z \text{ такие, что } a = bq + r \text{ и } 0 \leq r < |b|\]


\subsubsection{Доказательство существования}
Рассмотрим 3 случая.
\paragraph{Случай 1.}
Пусть $a \geq 0, b > 0$. Считая $b$ фиксированным (но любым), проведём индукцию по $a$.

\subparagraph{База индукции.}
$a < b$. Очевидно, что $a = b * 0 + a$, т. е. пара $q = 0, r = a$ такая, как требуется в формулировке теоремы.

\subparagraph{Индукционный переход.}
Пусть $a \geq b$. Тогда $0 \leq a-b < a$, т. е. пара $a-b, b$ удовлетворяет условиям теоремы ($b \neq 0$), условиям Случая 1 ($a-b \geq 0, b > 0$) и условиям индукционного предоположения ($a-b < a$). Поэтому $\exists q', r: a - b = bq' - r, 0 \leq r < |b|$, а тогда $a = b(q' + 1) + r)$, и пара $q = q' + 1$, $r$ - такая, как надо в формулировке теоремы.

\paragraph{Случай 2.}
Пусть $a < 0, b > 0$. Тогда $-a > 0$, и, как доказано в Случае 1, \[\exists q', r': -a = bq' + r', 0 \leq r' < |b| = b\]

Если $r' = 0$, то поскольку тогда $a = (-q')b$, пара $q = -q', r = 0$ такая, как надо. Если $r' \neq 0$, т. е. $r' > 0$, то, поскольку $a = b(-q' - 1) + (b - r')$ и $0 < b-r' < b = |b|$, пара $q = -q'-1, r = b - r'$ - такая, как надо.

\paragraph{Случай 3.}
Пусть $b < 0$ (при произвольном $a$). Тогда $-b > 0$, и, как показано в Случае 1 и Случае 2, существует $q', r$ такие, что 
\[a = (-b)q' + r = b(-q') + r \text{ и } 0 \leq r < -b = |b|\]
т. е. пара $q = -q', r$ такая, как надо.

\subsubsection{Доказательство единственности}
Пусть $a = bq + r = bq' + r'$, где $0 \leq r \leq r' < |b|$. Тогда $0 \leq r'-r < |b|$ и $b(q-q') = r'-r$, откуда $|b| * |q-q'| = |r' - r| = r' - r < |b|$, что возможно лишь при $q = q', r = r'$.

\subsection*{Список литературы}
Семёнов, Шмидт: стр. 57-58, 74


\newpage
\section{Наибольший общий делитель. Свойства (без алгоритма Евклида)}
\subsection{Определение: Наибольший общий делитель (пары, семейства)}
Пусть $(a, b)$ - пара чисел из $\mathbb Z$. \textbf{Наибольшим общим делителем} этой пары называется число $d \in \mathbb Z$ такое, что $d~\vert~a, d~\vert~b$ и $d \divs x$ для всякого $x \in \mathbb Z$ такого, что $x~\vert~a, x~\vert~b$.

Аналогично, наибольший общий делитель семейства - это общий делитель этого семейства, делящийся на любой его общий делитель.

\subsection{Предложение об ассоциированности НОД}
Пусть $d$ - наибольший общий делитель некоторой пары (семейства) в $\mathbb Z$. Число $d'$ есть наибольший общий делитель этой пары (семейства), когда $d \sim d'$, т. е. $d \divs d', d' \divs d$.

\subsubsection*{Доказательство}
\paragraph{Прямое.}
Если $d$ и $d'$ - наибольшие общие делители пары (семейства), то $d \divs d', d' \divs d$.

\paragraph{Обратное.}
Если $d \divs d'$, то $d'$ - общий делитель исходной пары (семейства). Если $d \divs d'$, то $d'$ делится и на любой общий делитель. Значит, $d'$ - наибольший общий делитель исходной пары (семейства).

\paragraph{Итог.}
Таким образом, если $n \geq 2$ и $a_1 = a_2 = \dots = a_n = 0$, то для чисел $a_1, \dots, a_n$ существует единственный наибольший общий делитель, равный $0$. Если $a_1, a_2, \dots, a_n$ все не равны $0$, то они имеют ровно два наибольших общих делителя, которые отличаются только знаком.

Множество всех наибольших общих делителей семейства (или пары) $(a_i)$ мы будем обозначать $\cgcd(a_i)$ или $\cgcd((a_i))$, а для пары $(a, b)$ - $\cgcd(a, b)$.

\subsection{Лемма}
\[\forall a, b, q \in \mathbb Z \quad \cgcd(a, b) = \cgcd(a + bq, b)\]
\subsubsection*{Доказательство}
Пусть $M$ - множество всех общих делителей пары $(a, b)$, а $M'$ - множество всех общих делителей пары $(a + bq, b)$. Тогда
\[x \in M \Leftrightarrow (a \divs x, b \divs x) \Leftrightarrow (a + bq \divs x, b \divs x) \Leftrightarrow x \in M'\]

\subsection{Свойства НОД}
\begin{enumerate}
\item $\cgcd(ca, cb) = c * \cgcd(a, b)$;
\item Если $a$ и $b$ взаимно просты, то $\cgcd(a, c) * \cgcd(b, c) = \cgcd(ab, c)$;
\item $\cgcd(a, 0) = a$;
\item Если $a \divs b$, то $\cgcd(a, b) = b$;
\item $\cgcd(ab, c) = \cgcd(b, c)$, если $a, c$ взаимно просты.
\end{enumerate}

\subsection*{Список литературы}
Семёнов, Шмидт: стр. 64-65

Глухов: стр. 59


\newpage
\section{Алгоритм Евклида. Линейное представление наибольшего общего делителя}
\subsection{Лемма}
\[\forall a, b, q \in \mathbb Z \quad \cgcd(a, b) = \cgcd(a + bq, b)\]
\subsubsection*{Доказательство}
Пусть $M$ - множество всех общих делителей пары $(a, b)$, а $M'$ - множество всех общих делителей пары $(a + bq, b)$. Тогда
\[x \in M \Leftrightarrow (a \divs x, b \divs x) \Leftrightarrow (a + bq \divs x, b \divs x) \Leftrightarrow x \in M'\]

\subsection{Алгоритм Евклида}
Пусть $(a, b)$ - пара чисел из $\mathbb Z$. Если $b = 0$, то $\cgcd(a, 0) = a$. Пусть $b \neq 0$.

Разделим с остатком $a$ на $b$:
\[a = bq_1 + r_1 \qquad 0 \leq r_1 < |b|\]

При этом $r_1 = a - bq_1$ и по лемме $\cgcd(a, b) = \cgcd(b, r_1)$, т. е. мы можем заменить пару $(a, b)$ парой $(b, r_1)$. Это и есть один шаг алгоритма. Делая цепочку целых чисел ${|b| > r_1 > r_2 > \dots \geq 0}$. Такая последовательность обязательно конечна, поэтому конечно и число шагов алгоритма. Очередной шаг невозможен лишь при условии, что при некотором $k$ мы получим $r_{k+1} = 0$, потому что только в этом случае $r_{k+1}$ невозможно разделить с остатком. Т. е. мы получим $(r_k, 0)$. Значит, $\cgcd(a, b) = \cgcd(r_k, 0) = r_k$. (Если уже $r_1 = 0$, то полагаем $r_0 = b$).

Таким образом, алгоритм Евклида можно представить такой схемой:
\begin{align*}
a &= bq_1 + r_1		\\
b &= r_1q_2 + r_2	\\
r_1 &= r_2q_3 + r_3	\\
\dots & \dots \dots \dots \dots		\\
r_{k-2} &= r_{k-1}q_k + r_k	\\
r_{k-1} &= r_kq_{k+1}
\end{align*}

\subsection{Теорема о линейном представлении наибольшего общего делителя}
У любого семейства $(a_i)_{i\in I}$ в $\mathbb Z$ существует наибольший общий делитель, и если $d = \cgcd(a_i)$, то в $\mathbb Z$ существует семейство $(u_i)_{i\in I}$ такое, что $d = \sum a_i u_i$.

\subsubsection*{Доказательство}
Если $\forall i \quad a_i = 0$, то $d = 0$ и можно взять $\forall i \quad u_i = 0$, так что утверждения теоремы очевидны.

Пусть в семействе $(a_i)$ есть ненулевые члены и пусть ${M = \left\{\sum a_i y_i : \forall i \quad y_i \in \mathbb Z \text{ и почти все } y_i = 0\right\}}$. Очевидно, что ${\forall i \quad a_i \in M}$ (можно взять $y_i = 1, \forall j \neq i \quad y_j = 0$) и в частности $M \neq \{0\}$.

Множество $\left\{ |x| : x \in M \setminus \{ 0 \} \right\}$ не пусто, состоит из целых чисел и ограничено снизу, поэтому в нём есть наименьшее число, которое есть абсолютная величина некоторого $d' \in M, d' \neq 0, d' = \sum a_i u'_i$ при некоторых $u'_i \in \mathbb Z$. Покажем, что $d'$ - наименьший общий делитель семейства $(a_i)$.

Для произвольного $j \in I$ разделим с остатком $a_j$ на $d'$: $a_j = d'q + r$. Тогда
\[r = a_j - d'q = a_j - q\sum a_i u'_i = (1 - u'_jq)a_j + \sum_{i \neq j}(-u'_i q)a_i \in M\]
и либо $r = 0$, либо $|r| < |d'|$.

Поскольку $r \in M$, а $|d'|$ - наименьшее значение абсолютной величины на $M$, неравенство $|r| < |d'|$ невозможно, поэтому $r = 0$ и $a_j  \divs  d'$. Таким образом, $d'$ - общий делитель семейства $(a_i)$.

Если $z$ - произвольный общий делитель $(a_i)$, то $d' = \sum a_i u'_i  \divs  z$, т. е. $d'$ - наибольший общий делитель семейства $(a_i)$. Существование наибольшего общего делителя доказано.

Если $d$ - любой наибольший общий делитель семейства $(a_i)$, то $d  \divs  d'$, поэтому \[d = \nu d', d = \nu\sum a_i u'_i = \sum a_i (\nu u'_i) = \sum a_i u_i, \text{ где } u_i = \nu u'_i \text.\]

\subsection*{Список литературы}
Семёнов, Шмидт: стр. 65-67

Глухов: стр. 60


\newpage
\section{Взаимно простые числа. Связь НОД и НОК. Наименьшее общее кратное взаимно простых чисел}

\subsection{Определение: Взаимная простота}
Пусть $a, b \in \mathbb Z$. Говорят, что $a$ и $b$ \textbf{взаимно просты}, если $\cgcd(a, b) = 1$.

Говорят, что семейство $(a)$ взаимно просто, если $\cgcd((a)) = 1$.

\subsection{Свойства взаимной простоты}

\subsubsection{}
Семейство $(a_i)_{i \in I}$ взаимно просто $\Leftrightarrow \exists (u_i)_{i \in I} \text{ в } \mathbb Z$ такое, что $\sum_{i \in I}a_i u_i = 1$.
\paragraph{Доказательство прямое.}
Это частный случай теоремы о линейном представлении НОД.
\paragraph{Доказательство обратное.}
Если $\forall i \in I \quad a_i  \divs d$ и $\sum a_i u_i = 1$, то $1  \divs  d$, поэтому $\cgcd((a_i)) = 1$.

\subsubsection{}
Если $d$ - НОД семейства $(a_i)$ и $\forall i \quad b_i = \frac{a_i}{d}$, то семейство $(b_i)$ взаимно просто.
\paragraph{Доказательство.}
По теореме о линейном представлении НОД $d = \sum a_i u_i = \sum d b_i u_i \Rightarrow 1 = \sum b_i u_i \Rightarrow (b_i)$ взаимно просто по п.1.

\subsubsection{}
Если каждый член конечного семейства $(a_i)$ взаимно прост с $b$, то и $\prod a_i$ взаимно просто с $b$.
\paragraph{Доказательство.}
По п.1 имеем $\forall i \quad a_i u_i + b \upsilon_i = 1$ для некоторых $u_i, \upsilon_i$. Тогда $a_i u_i = 1 - b \upsilon_i$; $(\prod a_i)u = 1 - b \upsilon$, где $u = \prod u_i, \upsilon$ - некоторое число. Итак, $(\prod a_i) u + b \upsilon = 1$ и по п.1 $\prod a_i$ и $b$ взаимно просты.

\subsubsection{}
Если $ab \divs c$ и $a, c$ взаимно просты, то $b \divs c$.
\paragraph{Доказательство.}
При некоторых $u, \upsilon \in \mathbb Z \quad au + b \upsilon = 1$. Умножая на $b$, получаем $b = b * 1 = abu + cb\upsilon$. Поскольку $ab \divs c$ и $cb\upsilon \divs c$, то и $b \divs c$.

\subsubsection{}
Если $a \divs b$, $a \divs c$ и $b, c$ взаимно просты, то $a \divs bc$.
\paragraph{Доказательство.}
$a \divs b \Rightarrow a = bx \divs c$, откуда по п.4 $x \divs c$, т. е. $x = cy$, и $a = bcy \divs bc$.

\subsection{Определение: Наименьшее общее кратное}
\textbf{Наименьшим общим кратным} целых чисел $a_1, a_2, \dots, a_n$ называется любое целое число $k$, удовлетворяющее следующим условиям:
\begin{enumerate}
\item[1)] $k$ есть общее кратное чисел $a_1, a_2, \dots, a_n$, т. е. $k \divs a_1, \dots, k \divs a_n$;
\item[2)] $k$ делит любое общее кратное чисел $a_1, a_2, \dots, a_n$, т. е. $\forall k_1 \in \mathbb Z \quad k_1 \divs a_1, \dots, k_1 \divs a_n \Rightarrow k~\vert~k_1$.
\end{enumerate}

Множество всех наименьших общих кратных чисел $a_1, \dots, a_n$ обозначают $\clcm\left\{a_1, \dots, a_n\right\}$.

Ясно, что если $n \geq 2$ и хотя бы одно из целых чисел равно $0$, то для них существует единственное наиименьшее общее кратное, равное $0$. Если же целые числа $a_1, \dots, a_n$ отличны от $0$, то для них существует ровно два наименьших общих кратных, отличающихся только знаком. Неотрицательное наименьшее общее кратное часто обозначают $\left[a_1, \dots, a_n\right]$.

Из канонического разложения чисел
\[n = \pm p_1^{\alpha_1} \dots p_k^{\alpha_k}, \quad m = \pm p_1^{\beta_1} \dots p_k^{\beta_k}\]
если условиться допускать нулевые показатели следует, что
\[(n, m) = p_1^{\gamma_1} \dots p_k^{\gamma_k}, \quad \left[n, m\right] = p_1^{\delta_1} \dots p_k^{\delta_k}\]
где $\gamma_i = \min(\alpha_i, \beta_i), \quad \delta_i = \max(\alpha_i, \beta_i), \quad i = 1, 2, \dots, k$.

При $n > 0, m > 0$ выполнено соотношение:
\[\cgcd(n, m) * \clcm(n, m) = nm\]

Если $n, m$ взаимно просты, то получаем
\[\clcm(n, m) = n * m\]

\subsection*{Список литературы}
Семёнов, Шмидт: стр. 67-69

Кострикин I: стр. 63

Глухов: стр. 62-63


\newpage
\section{Простые числа. Наименьший делитель числа. Бесконечность простых чисел}
\subsection{Определение: Простое число}
Натуральное число $p \neq 1$ называют \textbf{простым}, если оно не имеет натуральных делителей, отличных от $1$ и $p$, в противном случае оно называется составным. Число $1$ не относится ни к простым, ни к составным числам.

\subsection{Предложение о простоте наименьшего натурального делителя}
Для любого натурального числа $n > 1$ наименьший отличный от $1$ натуральный делитель всегда является простым числом.

\subsubsection*{Доказательство}
Рассмотрим множество $M$, состоящее из натуральных отличных от $1$ делителей числа $n$. Множество $M$ не пустое, так как $n \in M$. Значит, в множестве $M$ существует наименьшее число $q > 1$. \textit{Пусть} $q$ не простое, тогда существует $a$ такое, что $1 < a < q$ и $q \divs a$. Значит, $q$ не наименьшее в множестве $M$ - \textbf{противоречие}. Значит, $q$ - простое число.

\subsection{Теорема о бесконечности множества простых чисел}
\subsubsection*{Доказательство}
Предположим, что множество простых чисел конечно. Выписав их все в порядке возрастания, получим ряд чисел:
\begin{equation}
\label{eq:13_q}
2, 3, 5, \dots, p_r
\end{equation}

Рассмотрим число $N = (2 * 3 * \dots * p_r) + 1$. Так как каждое число из \eqref{eq:13_q} делит $2 * 3 * \dots * p_r$, но не делит $1$, то число $N$ не делится ни на одно из чисел \eqref{eq:13_q}, т. е. ни на одно простое число. А так как число $N$ больше единицы, то это противоречит основной теореме арифметики.

\subsection*{Список литературы}
Глухов: стр. 61-66


\newpage
\section{«Две» леммы о простых числах (НОД и деление произведения)}
\subsection{Лемма о наибольшем общем делителе}
Для любых целых чисел $a, b, q \in \mathbb Z$
\[\cgcd(a, b) = \cgcd(a + bq, b)\]

\subsubsection*{Доказательство}
Пусть $M$ и $M'$ - множества всех общих делителей пар $(a, b)$ и $(a + bq, b)$ соответственно. Тогда
\[x \in M \Leftrightarrow (a \divs x, b \divs x) \Leftrightarrow (a + bq \divs x, b \divs x) \Leftrightarrow x \in M'\]
т. е. $M = M'$. Отсюда следует требуемое.

\subsection{Лемма о делении произведения}
Если $ab \divs c \quad (a, b, c \in \mathbb Z)$ и $a, c$ взаимно просты, то $b \divs c$.
\subsubsection*{Доказательство}
При некоторых $u, \upsilon \in \mathbb Z \quad au = c \upsilon = 1$. Умножая это равенство на $b$, имеем:
\[b = b * 1 = b(au + c \upsilon) = abu + cb \upsilon\]

Поскольку $ab \divs c$ и $cb \upsilon \divs c$, то и $b \divs c$.

\subsection{Лемма о делении произведения (ещё одна)}
Если $ab \divs p \quad (a, b, p \in \mathbb Z)$ и $p$ простое, то $a \divs p$ или $b \divs p$.
\subsubsection*{Доказательство}
Если $a$ не делится на $p$, то $a$ и $p$ взаимно просты, а тогда по предыдущей лемме $b \divs p$. Таким образом, $a \divs p \vee b \divs p$.

\subsection*{Список литературы}
Семёнов, Шмидт: стр. 65, 69, 70



\newpage
\section{Основная теорема арифметики. Каноническое разложение наибольшего общего делителя и наименьшего общего кратного}
\subsection{Основная теорема арифметики}
Каждое положительное целое число $n \neq 1$ может быть записано в виде произведения простых чисел:
\begin{equation}
\label{eq:15_1}
n = p_1 p_2 \dots p_s
\end{equation}
Эта запись единственна с точностью до порядка множителей.
\subsubsection{Доказательство существования}
Так как $2$ - простое число, то для $n = 2$ утверждение теоремы верно. Допустим, что оно верно для всех целых положительных чисел $n: 2 \leq n \leq m$ при любом фиксированном натуральном числе $m \geq 2$. Докажем существование разложения \eqref{eq:15_1} для $m+1$.

Пусть $m+1$ не простое число. Тогда оно делится на некоторое число $a$ такое, что $1 < a < m+1$. Следовательно, $m + 1 = a * b$, где $1 < b < m+1$.

По предположению индукции каждое из чисел $a, b$ разлагается в произведение простых чисел:
\[a = p_1 \dots p_k, \quad b = q_1 \dots q_l\]
где $p_1, \dots, p_k, q_1, \dots, q_l$ - простые числа, $k, l \in \mathbb N$.

Отсюда $m + 1 = p_1 \dots p_k q_1 \dots q_e$.
\subsubsection{Доказательство единственности}
Пусть $n = p_1 \dots p_r = p'_1 \dots p'_{r'}$ и все $p_i, p'_j$ просты. Пусть для определённости $r \leq r'$.
\paragraph{База индукции: $r = 1$.}
$p_1 = p'_1 \dots p'_{r'} \divs p'_1$. Тогда $p'_1 = p_1, r' = 1$.
\paragraph{Индукционный переход.}
Пусть $r \geq 2$. $p'_1 \dots p'_{r'} \divs p_r$. Тогда $\exists k : p'_k = p_r$. Перенумеровав, если надо, сомножители, можем считать, что $k = r'$. Сокращая исходное равенство на $p_r$, получаем $p_1 \dots p_{r-1} = p'_1 \dots p'_{r' - 1}$, т. е. $r = r'$ и $p_i = p'_i$, $i = 1, 2, \dots, r-1$.

\subsection{Определение: разложение на простые множители}
Указанное в теореме разложение называют \textbf{разложением на простые множители}.

\subsection{Определение: каноническое разложение}
Перенумеровав сомножители так, чтобы они были попарно различны, мы получим
\[n = p_1^{k_1} p_2^{k_2} \dots p_m^{k_m}\]

Такое представление называется \textbf{каноническим разложением} числа $n$.

\subsection{Каноническое разложение наибольшего общего делителя и наименьшего общего кратного}
Любые два целых числа можно записать в виде произведения степеней одних и тех же простых чисел,
\[n = \pm p_1^{\alpha_1} \dots p_k^{\alpha_k}, \quad m = \pm p_1^{\beta_1} \dots p_k^{\beta_k}\]
если условиться допускать нулевые показатели, считая $p_i^0 = 1$. Тогда, очевидно,
\begin{equation}
\label{eq:15_2}
\begin{array}{ll}
\cgcd(n, m) &= p_1^{\gamma_1} \dots p_k^{\gamma_k} \\
\clcm(n, m) &= p_1^{\delta_1} \dots p_k^{\delta_k}
\end{array}
\end{equation}
где $\gamma_i = \min(\alpha_i, \beta_i), \quad \delta_i = \max(\alpha_i, \beta_i), \quad i = 1, 2, \dots, k$.

Из \eqref{eq:15_2} вытекают следующие утверждения:
\begin{enumerate}
\item[1)] $\cgcd(n, m)~\vert~n, \cgcd(n, m)~\vert~m$, и если $k~\vert~n, k~\vert~m$, то $k~\vert~\cgcd(n, m)$.
\item[2)] $n~\vert~\clcm(n, m), m~\vert~\clcm(n, m)$, и если $n~\vert~u, m~\vert~u$, то $\clcm(n, m)~\vert~u$.
\end{enumerate}

При $n > 0, m > 0$ выполнено соотношение:
\[\cgcd(n, m) * \clcm(n, m) = nm\]

Если $n, m$ взаимно просты, то $\clcm(n, m) = n * m$.

\subsection*{Список литературы}
Кострикин I: стр. 62-63

Семёнов, Шмидт: стр. 71-73

Глухов: стр. 26, 64-65



\newpage
\section{Непрерывные дроби. Вычисление подходящих дробей}
\subsection{Определение: Числовая непрерывная дробь}
\textbf{Числовой непрерывной дробью} называется выражение
\begin{equation}
\label{eq:16_1}
\frac{a_1}{b_1 + \frac{a_2}{b_2 + \dots}}
\end{equation}
где $a_1, b_1, a_2, b_2, \dots$ - некоторые отличные от нуля числа.

\subsection{Определение: Конечная/бесконечная непрерывная дробь}
Непрерывная дробь называется \textbf{конечной}, если она имеет лишь конечное число звеньев. В противном случае она называется \textbf{бесконечной}.

\subsection{Определение: Подходящая дробь}
$n$-й \textbf{подходящей дробью непрерывной дроби} \eqref{eq:16_1} называется конечная дробь
\[\frac{a_1}{b_1 + \frac{a_2}{b_2 + \dots + \frac{a_n}{b_n}}}\]
получаемая из непрерывной дроби \eqref{eq:16_1} её обрываением на $n$-ом шаге.

В этом определении предполагается, что число звеньев в непрерывной дроби \eqref{eq:16_1} не меньше $n$.
\subsubsection*{Пример}
Дробь $\frac{10}{7}$. Наибольшее целое число, не превосходящее эту дробь - это $1$.
\[\frac{10}{7} = 1 + \frac{3}{7}\]
Перевернём дробь $\frac{3}{7}$:
\[\frac{10}{7} = 1 + \frac{1}{\frac{7}{3}}\]
Наибольшее целое число, не превосходящее $\frac{7}{3}$ - это $2$.
\[\frac{10}{7} = 1 + \frac{1}{2 + \frac{1}{3}} \text{ - непрерывная дробь для } \frac{10}{7}.\]

\subsection{Композиция дробно-линейных преобразований}
Пусть $(S_1 \circ \dots \circ S_n)w$ - композиция дробно-линейных преобразований.
\[S_n(w) = \frac{a_n}{b_n + w}, \quad n = 1, 2, \dots\]
Ясно, что $(S_1 \circ S_n)(0)$ равно значению $n$-й подходящей дроби непрерывной дроби \eqref{eq:16_1}.

\subsubsection*{Примеры}
\[\sqrt{2} = 1 + (\sqrt{2} - 1) = 1 + \frac{1}{\sqrt{2} + 1} = 1 + \frac{1}{2 + \sqrt{2} - 1} = 1 + \frac{1}{2 + \frac{1}{\sqrt{2} + 1}} = \dots = \left[1; 2, 2, 2, \dots\right]\]
\[\frac{\sqrt{5} - 1}{2} = 0 + \frac{1}{\frac{2}{\sqrt{5} - 1}} = \frac{1}{\frac{\not 2 (\sqrt{5} + 1)}{\not 4 {}_2}} = \frac{1}{1 + \frac{\sqrt{5}+1}{2} - 1} = \frac{1}{1 + \frac{\sqrt{5} - 1}{2}} = \frac{1}{1 + \frac{1}{\frac{2}{\sqrt{5} - 1}}} = \frac{1}{1 + \frac{1}{\frac{\not 2 (\sqrt{5} + 1)}{\not 4 {}_2}}} =\]
\[= \frac{1}{1 + \frac{1}{1 + \frac{\sqrt{5}+1}{2} - 1}} = \frac{1}{1 + \frac{1}{1 + \frac{\sqrt{5} - 1}{2}}} = \dots = \left[0; 1, 1, 1, \dots\right] \text{ (золотое сечение)}\]

\subsection{Теорема}
Рациональному числу $\frac{p}{q}$ ставится в соответствие конечная непрерывная дробь $[b_0; b_1, b_2, \dots, b_n]$. Иррациональному числу $a$ ставится в соответствие бесконечная непрерывная дробь $[b_0; b_1, b_2, \dots, b_n, \dots]$. При этом знак соответствия в случае рационального числа можно заменить знаком равенства. ($\alpha \sim [b_0; b_1, b_2, \dots]; b_0 \in \mathbb Z$).

(Всякому положительному вещественному числу $\alpha$ ставится в соответствие непрерывная дробь с натуральными коэффициентами по следующему алгоритму: $\alpha = [\alpha] + \{\alpha\} = [\alpha] + \frac{1}{\frac{1}{\{\alpha\}}} = \dots$).


\newpage
\section{Сравнения. Свойства сравнений}
Пусть $m$ - фиксированное натуральное число, $m > 1$. Множество $m \mathbb Z$, очевидно, замкнуто не только относительно операции сложения, но и относительно операции умножения и удовлетворяет всем аксиомам кольца.
\subsection{Определение: Сравнимые числа, модуль сравнения}
Два целых числа $n, n'$ называются \textbf{сравнимыми} по модулю $m$, если при делении на $m$ они дают одинаковые остатки. При этом пишут $n \equiv n' \enspace (m)$ или $n \equiv n' \pmod{m}$, а число $m$ называют \textbf{модулем сравнения}.

\subsection{Свойства сравнений}
\begin{enumerate}
\item $ac \equiv bc \pmod{m} \Leftrightarrow a \equiv b \pmod{\frac{m}{\cgcd(c, m)}}$,

в частности, если $c, m$ взаимно просты, то $ac \equiv bc \pmod{m} \Leftrightarrow a \equiv b \pmod{m}$.
\paragraph{Доказательство.}
Пусть $d = \cgcd(c, m), c = dc_1, m = dm_1$, тогда $c_1, m_1$ взаимно просты.
\[ac \equiv bc \pmod{m} \Leftrightarrow ac - bc \divs m \Leftrightarrow (a-b) c = mx \Leftrightarrow (a-b)\not d c_1 = \not d m_1 x \divs m_1 \Leftrightarrow\]
\[\Leftrightarrow a-b \divs m_1 \Leftrightarrow a \equiv b \pmod{m_1} \text{, где } m_1 = \frac{m}{\cgcd(c, m)}\]

\item Если $m, n$ взаимно просты, то
\[a \equiv a' \pmod{mn} \Leftrightarrow 
\begin{cases}
a \equiv a' \pmod{m} \\
a \equiv a' \pmod{n}
\end{cases}
\]

\paragraph{Доказательство прямое.}
Очевидно.
\paragraph{Доказательство обратное.}
Прямо следует из свойства взаимной простоты: если $a \divs m, a \divs n$ и $m, n$ взаимно просты, то $a \divs mn$.
\end{enumerate}

\subsection*{Список литературы}
Шмидт: стр. 210, 173

Кострикин I: стр. 155


\newpage
\section{Полная система вычетов. Кольцо классов вычетов (корректность операций)}
\subsection{Определение: Сравнимые числа}
Пусть $a, a', m \in \mathbb Z; m > 1$. Говорят, что $a$ \textbf{сравнимо} с $a'$ по модулю $m$ (и пишут $a \equiv a' \pmod{m}$, если $a - a' \divs m$. Запись $a \equiv a' \pmod{m}$ называют сравнением (по модулю $m$).

\subsection{Определение: Сравнимость, класс вычетов}
Данное определение для каждого фиксированного $m$ задаёт в множестве $\mathbb Z$ бинарное отношение - \textbf{сравнимость} (по модулю $m$). Ясно, что сравнимость рефлексивна, симметрична и транзитивна и, значит, разбивает $\mathbb Z$ на классы эквивалентности. Эти классы называются \textbf{классами вычетов} (по модулю $m$). В каждом классе сравнимости по модулю $m$ можно выбрать представителя.

\subsection{Определение: Полная система вычетов}
Множество представителей, выбранных по одному из каждого класса вычетов, называют \textbf{полной системой вычетов} (по данному модулю).

\subsection{Вывод кольца}
Если $a \equiv a', b \equiv b'$, то $a \pm b \equiv a' \pm b', ab \equiv a'b'$ (сравнения по одному модулю можно почленно складывать, вычитать и перемножать). Доказательство очевидно. Другими словами, сравнимость согласована со сложением, вычитанием и умножением, т.е. является конгруэнцией в кольце $\mathbb Z$.

Это позволяет перенести действия кольца на множество классов конгруэнции, т.е. классов вычетов.

Эти действия определяются равенствами:
\[\overline a + \overline b = \overline{a + b}\]
\[\bar a + \bar b = \overline{ab}\]


Проверим, что класс-результат не зависит от выбора представителей классов-операндов.
\[
\left.
\begin{aligned}
\overline a &= \overline{a'} &\Rightarrow a &\equiv a' \\
\overline b &= \overline{b'} &\Rightarrow b &\equiv b'
\end{aligned}
\right\} \Rightarrow \left\{
\begin{aligned}
a + b &\equiv a' + b'   &\Rightarrow   \overline{a + b} &= \overline{a' + b'} \\
ab &\equiv a'b'   &\Rightarrow ~\quad \overline{ab} &= \overline{a'b'}
\end{aligned}
\right.
\]

Ясно, что эти действия в множестве классов вычетов всюду определены.

Ясно также, что все аксиомы кольца выполняются для множества классов вычетов:
\begin{itemize}
\item $\overline 0$ - нуль сложения классов
\item $\overline {-a} = - \overline a$
\item $\overline 1$ - единица сложения классов
\end{itemize}

Таким образом, введённые нами действия задают на множестве классов вычетов по модулю $m$ структуру коммутативного кольца с единицей. Это кольцо называется \textbf{кольцом вычетов} (или \textbf{кольцом классов вычетов}) по модулю $m$ и обозначается $\mathbb Z_m$ (или $\mathbb Z/m \mathbb Z$, $\mathbb Z/(m)$).

\subsection*{Список литературы}
Шмидт: стр. 210
Семёнов, Шмидт: стр. 75-78


\newpage
\section{Приведённая система вычетов. Функция Эйлера. Вычисление функции Эйлера (без доказательства мультипликативности).}
\subsection{Определение: Приведённая система вычетов}
Группа $(\mathbb Z/m \mathbb Z)^*$ состоит из всех тех классов вычетов, чьи представители взаимно просты с модулем $m$; мы будем говорить, что такой класс взаимно прост с модулем.

Множество представлений, выбранных по одному из каждого такого класса, называют \textbf{приведённой системой вычетов} (по данному модулю). Таким образом, $\{a_1, a_2, \dots \}$ тогда и только тогда есть приведённая система вычетов по модулю $m$, когда $\{\overline{a_1}, \overline{a_2}, \dots\} = (\mathbb Z/m \mathbb Z)^*$.

\subsection{Определение: Функция Эйлера}
\textbf{Функцией Эйлера} называется отображение $\varphi: \mathbb N \rightarrow \mathbb N$ такое, что $\forall a \in \mathbb N \quad \varphi(a)$ есть количество натуральных чисел, не превосходящих $a$ и взаимно простых с $a$.

Функция Эйлера мультипликативна, т.е. для любой взаимно простой пары \[a, b \in \mathbb N \quad \varphi(ab) = \varphi(a)\varphi(b)\]

Если $n = p_1^{k_1} \dots p_m^{k_m}$ - каноническое разложение числа $n \in \mathbb N$, то
\[ \varphi(n) = p_1^{k_1-1}(p_1 - 1) \dots p_m^{k_m - 1}(p_m - 1) \]

Если $p \in \mathbb N$ просто, то не взаимно просты с $p$ те и только те числа, которые кратны $p$; ясно, что в интервале $[1, p^k]$ имеется $p^{k-1}$ таких чисел, и поэтому
\[ \varphi(p^k) = p^k - p^{k-1} = p^{k-1}(p-1) \]

\subsection*{Список литературы}
Шмидт: стр. 211-212


\newpage
\section{Система сравнений. Китайская теорема об остатках}
\subsection{Китайская теорема об остатках}
Пусть $(m_i)_{i \in I}$ - конечное семейство попарно простых натуральных чисел. Тогда для любого семейства $(a_i)_{i \in I}$ существует $x \in \mathbb Z$ такое, что 
\begin{align*}
x & \equiv a_1 \pmod{m_1} \\
x & \equiv a_2 \pmod{m_2} \\
&\dots \dots \dots \dots \dots \\
x & \equiv a_k \pmod{m_k}
\end{align*}
($|I| = k$) и это решение единственно по модулю $m_1 m_2 \dots m_k$.

\subsubsection*{Доказательство}
Индукция по $k$. При $k = 1$ утверждение верно. Пусть $k > 1$. По предположению индукции система, составленная из первых $k-1$ сравнений имеет единственное по модулю $m' = m_1 m_2 \dots m_{k-1}$. Пусть это \[x \equiv a \pmod{m'}\]

Тогда класс вычетов $a\pmod{m'}$ - это множество всех чисел вида \begin{equation}
\label{eq:20_1}
x = a + m'y, \quad y \in \mathbb Z
\end{equation}

Для нахождения всех решений исходной системы сравнений осталось найти все те значения $y$, при которых числа вида \eqref{eq:20_1} удовлетворяют последнему сравнению исходной системы. Подставим вместо $x$ его выражение \eqref{eq:20_1} и решим полученное сравнение относительно $y$.
\[a + m'y \equiv a_k \pmod{m_k}\]
\[m'y \equiv a_k - a \pmod{m_k}\]

Так как $m', m_k$ взаимно просты, сравнение имеет единственное решение по модулю $m_k$ (так как $m_k$ и $m$ взаимно просты). Пусть это будет класс
\[\{b + m_k t : t \in \mathbb Z \} \]

Подставляя в \eqref{eq:20_1}, получаем:
\[a + m'(b + m_k t) = a + m'b + m' m_k t\]

Итак, множество решений исходной системы сравнений совпадает с классом \[a + b m' \pmod{m_1 m_2 \dots m_k}\]

\subsection{Замечание}
Из доказательства китайской теоремы об остатках виден и алгоритм решения системы сравнений.
\begin{enumerate}
\item Из первого сравнения находим $x = a_1 + m_1 y$.
\item Подставив $x$ во второе и решив полученное сравнение относительно $y$, получим
\[y = b_1 + m_2z \Rightarrow x = a_1 + m_1b_1 + m_1m_2z\]
\item Подставив найденные решения $x$ в третье сравнение системы, находим $z$ и т.д.
\end{enumerate}

\subsection{Пример}
\[ax \equiv b \pmod{m}\]

$a, m$ взаимно просты.

\subsubsection*{Единственность решения}
Пусть
\[x_1 \equiv b_1 \pmod{m} \]
\[x_2 \equiv b_2 \pmod{m} \]

И то, и другое - решения. Значит,

\begin{align*}
ax_1 &\equiv b \pmod{m} \\
ax_2 &\equiv b \pmod{m} \\
a(x_1 - x_2) &\equiv 0 \pmod{m} \quad \vert * a^{-1} \pmod{m} \\
x_1 - x_2 &\equiv 0 \pmod{m} \\
\end{align*}

Более строго:
\begin{align*}
ab_1 &\equiv b \pmod{m} \\
ab_2 &\equiv b \pmod{m} \\
a(b_1 - b_2) &\equiv 0 \pmod{m} \\
a & \not\equiv 0 \pmod{m} \text{ или умножим на } a^{-1} \\
b_1 &\equiv b_2 \pmod{m}
\end{align*}

\subsection*{Список литературы}
Шмидт: стр. 180

Глухов: стр. 86-87



\newpage
\section{Мультипликативность функции Эйлера}
Функция Эйлера мультипликативна, т.е. для любой взаимно простой пары \[a, b \in \mathbb N \quad \varphi(ab) = \varphi(a) \varphi(b)\]

\subsection{Доказательство}
Пусть $a_1, a_2, \dots, a_{\varphi(a)}$ и $b_1, b_2, \dots, b_{\varphi(b)}$ - приведённые системы вычетов по модулям $a$ и $b$ соответственно.

По Китайской теореме об остатках
\[\forall i, j \quad \exists c_{ij} \begin{cases}
c_{ij} \equiv a_i \pmod{a} \\
c_{ij} \equiv b_i \pmod{b}
\end{cases}\]

Ясно, что $c_{ij}$ взаимно просто с $a$ и с $b$ (ибо таковы соответственно $a_i$ и $b_j$), а значит и с $ab$ (свойство взаимной простоты: если $a_i, c$ взаимно просты, то и $\prod a_i$ и $c$ взаимно просты).

Все $c_{ij}$ лежат в разных классах по модулю $ab$ так как любая пара различных $c_{ij}$ не сравгтма хотя бы по одному из модулей $a, b$ (например, если $i \neq k$, то $c_{ij} \equiv a_i \not \equiv a_k \equiv c_kl \pmod{a}$).

Если $x$ взаимно просто с $ab$, то $x$ взаимно просто и с $a$, и с $b$, а потому \[\exists c_{ij} : x \equiv a_i \pmod{a} \quad x \equiv b_i \pmod{b}\]

Тогда $x \equiv c_{ij} \pmod{ab}$ (свойство взаимной простоты: если $a \divs b, a \divs b$, $b$ и $c$ взаимно просты, то $a \divs bc$).

Таким образом, числа $c_{ij}$ представляют все классы по модулю $ab$, взаимно простые с этим модулем (каждый по одному разу), и, значит, количество таких классов равно количеству этих чисел, т.е. $\varphi(ab) = \varphi(a)\varphi(b)$.

\subsection*{Список литературы}
Шмидт: стр. 212


\newpage
\section{Теорема Эйлера. Малая теорема Ферма}
\subsection{Теорема Эйлера}
Если $a \in \mathbb Z$ и $m \in \mathbb N$ взаимно просты, то
\[a^{\varphi(m)} \equiv 1 \pmod{m} \Leftrightarrow \overline{a}^{\varphi(m)} = \overline 1 \text{ в } \mathbb Z /m \mathbb Z\]

\subsubsection*{Доказательство}
Пусть \[(\mathbb Z/m \mathbb Z)^* = \{\overline{a}_1, \overline{a}_2, \dots, \overline{a}_{\varphi(m)}\}\] Ясно, что \[\overline{a}(\mathbb Z/m \mathbb Z)^* = (\mathbb Z/m \mathbb Z)^*\] поэтому \[\prod \overline{a_i} = \prod_{i=1}^{\varphi(m)} \bar{a} \bar{a_i} = \overline{a}^{\varphi(m)} \prod \overline{a_i}\] откуда, сокращая на $\prod \overline{a_i}$, получаем \[\overline{1} = \overline{a}^{\varphi(m)}\]

\subsection{Следствие (малая теорема Ферма)}
Пусть, $p \in \mathbb N, a \in \mathbb Z$. Если, $p$ просто и $a \not \divs p$, то $a^{p-1} \equiv 1 \pmod{p}$, или, что то же, $\overline{a}^{p-1} = \overline{1} \text{ в } F_p$.

\subsubsection*{Доказательство}
Это частный случай теоремы Эйлера. Если $p$ просто и $a \not \divs p$, то $a, p$ взаимно просты. Так как $p$ просто, $\varphi(p) = p-1$.

\subsection*{Список литературы}
Шмидт: стр. 213


\newpage
\section{Примечания редактора}
Этот документ был перепечатан с рукописных конспектов Королёва Алексея Васильевича, за предоставление которых невозможно полномерно выразить благодарность.

В случаях «очевидного» доказательства читателю рекомендуется как минимум пробежаться в голове о том, как это доказывается; в случаях недопониманий материала рекомендуется обращаться к первоисточникам, указанным в конце каждого билета. Не забывайте: в случае несдачи Вам придётся отвечать на один из этих билетов вдобавок к 38 билетам по алгебре на сессию и другим предметам!

Полные названия первоисточников, указанных в конце каждого билета:
\begin{enumerate}
\item «Шмидт»: Шмидт Р. А. --- Алгебра: Учеб. пособие. Ч. 1. --- СПб: Изд-во С.-Петерб. ун-та, 2008 --- 360 с.; ISBN 978-5-288-04515-8
\item «Семёнов, Шмидт»: А.А. Семенов, Р.А. Шмидт --- Начала алгебры. Часть I. --- СПб.: НИИХимии СПбГУ, 2002. --- 156 с.; ISBN не имеет
\item «Кострикин I»: Кострикин А. И. Введение в алгебру. Часть I. Основы алгебры: Учебник для вузов. М.: ФИЗМАТЛИТ (точное издание мне неизвестно).
\item «Глухов»: \textit{Скорее всего} Глухов М. М., Елизаров В. П., Нечаев А. А., Алгебра: Учебник --- 2-е изд. испр. и доп. (3+ изд., стер.) --- СПб.: \textbf{Издательство «Лань»}, 2015+ --- 608 с.: ил.; ISBN 978-5-8114-1961-6; ISBN 978-5-8114-4775-6
\end{enumerate}

Этот документ в основном писался в автобусах и поездах, и, хоть он и был мной перечитан перед публикацией, он может содержать некоторые типографические недочёты, опечатки или неточности. При нахождении последних обращайтесь к первоисточнику или профессору за советом, а потом к редактору; при нахождении типографических недочётов или не изменяющих смысл опечаток просто обращайтесь к редактору: @alexmush в Telegram.

Удачи Вам на коллоквиуме!

\hspace*{\fill} - Алекс Машкович

\end{document}
